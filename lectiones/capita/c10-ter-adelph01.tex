
\poemtitle*{Scaena I. Micio.}


\settowidth{\versewidth}{aliquid. vah quemquamne hominem in animo instituere aut}
\begin{verse}[\versewidth]\poemlines{5}
\setverselinenums{26}{30}
%{\large

\textit{Micio:} Storax! – non rediit hac nocte a cena Aeschinus\\
    neque servolorum quisquam qui advorsum ierant.\\
    profecto hoc vere dicunt: si absis uspiam\\
    aut ibi si cesses, evenire ea satius est\\
    quae in te uxor dicit et quae in animo cogitat\\
    irata quam illa quae parentes propitii.\\
    uxor, si cesses, aut te amare cogitat\\
    aut tete amari aut potare atque animo obsequi\\
    et tibi bene esse soli, quom sibi sit male.\\
    ego quia non rediit filius quae cogito et\\
    quibu' nunc sollicitor rebu'! ne aut ille alserit\\
    aut uspiam ceciderit aut praefregerit\\
    aliquid. vah quemquamne hominem in animo instituere aut\\
    parare quod sit carius quam ipsest sibi!\\
    atque ex mě hĭc natu' non est sed ěx fratre. is adeo\\
    dissimili studiost iam inde ab adulescentia:\\
    egŏ hanc clementem vitam urbanam atque otium\\
    secutu' sum et, quod fortunatum isti putant,\\
    uxorem, numquam habui. ille contra haec omnia:\\
    ruri agere vitam; semper parce ac duriter\\
    se habere; uxorem duxit; nati filii\\
    duo; inde ego hunc maiorem adoptavi mihi;\\
    eduxi a parvolo; habui amavi pro meo;\\
    in eo me oblecto, solum id est carum mihi.\\
    ille ut item contra me habeat facio sedulo:\\
    do praetermitto, non necesse habeo omnia\\
    pro meo iure agere; postremo, alii clanculum\\
    patres quae faciunt, quae fert adulescentia,\\
    ea ne me celet consuefeci filium.\\
    nam qui mentiri aut fallere institerit patrem aut\\
    audebit, tanto magis audebit ceteros.\\
    pudore et liberalitate liberos\\
    retinere satius esse credo quam metu.\\
    haec fratri mecum non conveniunt neque placent.\\
    venit ad me saepe clamităns ``quid agi', Micio?\\
    quor perdis adulescentem nobis? quor amat?\\
    quor potat? quor tu his rebu' sumptum suggeris,\\
    vestitu nimio indulges? nimium ineptus es.''\\
    nimium ipse durust praeter aequomque et bonum,\\
    et errat longe meă quidem sententia\\
    qui imperium credat gravius esse aut stabilius\\
    vi quod fit quam illud quod amicitia adiungitur.\\
    mea sic est ratio et sic animum induco meum:\\
    malo coactu' qui suom officium facit,\\
    dum id rescitum iri credit, tantisper cavet;\\
    si sperat fore clam, rursum ad ingenium redit.\\
    ill' quem beneficio adiungas ex animo facit,\\
    studět par referre, praesens absensque idem erit.\\
    hoc patriumst, potiu' consuefacere filium\\
    suă sponte recte facere quam alieno metu:\\
    hoc pater ac dominus interest. hoc qui nequit\\
    fateatur nescire imperare liberis.\\
    sed estne hic ipsu' de quo agebam? et certe is est.\\
    nescioquid tristem video: credo, iam ut solet\\
    iurgabit. salvom te advenire, Demea,\\
    gaudemus.\\
%}
\end{verse}

%\newpage

\section*{Parafraza}

\textsc{Mi.} Heus, Storax; non reversus est domum Aeschinus filius hac nocte a convivio, cui affuit cum sodalibus suis; neque ullus ex servis, qui obviam ei missi fuerant eum comitatum. Equidem haec communis sententia vera est: minus damnosum est ea accidere quae uxor irata de te suspicatur, et animo conjicit, quam illa, quae parentes pii et filiorum studiosi eis timent. Uxor, si tardes domum redire, aut alius mulieris amore detineri te suspicatur, aut aliam amore tui captam esse, aut compotationibus te deditum esse, aut voluptati inservire ; denique te solum in deliciis versari, dum ipsa molestiis et tristitia cruciatur. Ah, quae suspiciones mentem meam agitant, quia non rediit filius! quibus timoribus animus angitur! ne aut ille Aeschinus algore obstupuerit, aut uspiam lapsus membrum aliquod sibi omnino ruperit. Vah! potestne illud esse, ut quispiam ita animo sit affectus, ut quicquam carius habeat quam seipsum ? Attamen ex me non est genitus, sed ex fratre meo Demea. Is frater dissimiliter vitam instituit, atque ego: jam inde usque ab ephebis ego hanc comem elegi vitam, urbanam et tranquillam; et, quod plerique beatum aestimant, uxorem nunquam ducere volui: ille omnia alia potiora duxit, agro colendo vitam exercere, semper austere et laboriose se tractare: uxorem matrimonio sibi junxit: filios duos genuit: deinde hunc ego majorem natu in adoptionem accepi: a puero educari, habui, et dilexi tanquam filium naturalem; in eo delicias omnes capio, hoc solum pretiosum habeo: ego omni cura laboro, ut pariter me carum vicissim habeat ille: do sumtus, praetermitto et dissimulo delicta ejus, nec oportere me puto in omnibus rebus patria potestate uti erga illum. Postremo, ita eum institui, ne mihi occultet ea, quae alii filii clam patribus suis faciunt, nempe culpas, quibus obnoxia est aetas adolescentiae; nam, qui mendacium proloqui coram patre, aut eum solitus fuerit verbis decipere, multo audacius haec faciet in ceteros. Utilius et melius esse censeo, verecundia et liberali clementia eum avocare a vitiis, quam timore. Haec non ita fratri, ut mihi, videntur observanda, neque probantur. Ille saepe me convenit vociferans: Quid facis, Micio? cur nobis corrumpis filium? cur sumtus das ei, ut amicas habeat? ut in potationibus se mergat? cur vestimenta splendidiora et nimio pretiosiora largiris? nimis imprudenter eum moderaris. Ille vero (Demea) asperior est, et saevior plus aequo et bono; et valde fallitur ita sentiendo, ut ego quidem opinor, quicumque sibi persuadet imperium, quod cum violentia quaeritur et exercetur, esse fortius et diuturnius, quam illud quod amore conciliatur. Ego sic reor, credoque; qui vi aut necessitate adigitur ad suum faciendum officium, tantisper cavet sibi, et abstinet a malo, dum id cognitum iri putat: si credit occultum fore, rursus ad suos mores revertitur. Ille, quem beneficio acquiras, omni studio libens officium facit, curat diligenter vicissim benefacere; sive adsit coram, vel clam absit, similem sui semper se geret, et in officio suo manebit. Hoc patrem decet potius assuefacere filium, libenter et ultro recte faciendo, quam terrendo exemplis propositis poenarum alienarum. Haec differentia est inter patrem ac dominum. Qui non potest talem se praestare, concedat hic oportet se ignarum esse regendi liberorum ingenia. Sed estne hic ipse Demea, de quo loquebar? Equidem is ipse est. Nescio propter quid tristis mihi videtur: puto jam pro more suo jurgiis mecum aget. Salvum te adesse laetor, Demea.


