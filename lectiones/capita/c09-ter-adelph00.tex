\section*{Didascalia}

{\large

\noindent INCIPIT TERENTI ADELPHOE : ACTA LVDIS FVNEBRIBVS L · AEMELIO PAVLO QVOS FECERE Q · FABIVS MAXVMVS P · CORNELIVS AFRICANVS : EGERE L · AMBIVIVS TVRPIO L · HATILIVS PRAENESTINVS : MODOS FECIT FLACCVS CLAVDI TIBIIS SARRANIS TOTA : GRAECA MENANDRV : FACTA VI M · CORNELIO CETHEGO L · ANICIO GALLO COS ·\\

}


\noindent Adelphi, Comoedia Terentii, acta ludis funebribus Lucii Aemilii Pauli, quos dederunt Q.\ Fabius Maximus, P.\ Cornelius Africanus, illius filii. Egerunt L.\ Attilius Pranestinus et Minucius Protimus. Modos fecit et cantica Flaccus Claudii filius, tibiis Sarranis paribus, sinistris. Haec Comoedia facta est Latina e Graeca Menandri, L.\ Anicio Gallo, M.\ Cornelio Cethego Consulibus.

\poemtitle*{C.\ Sulpici Apollinaris periocha}

\settowidth{\versewidth}{sed Ctesiphonem retinet. hunc citharistriae}
\begin{verse}[\versewidth]\poemlines{5}
{\large
    Duos cum haberet Demea adulescentulos,\\
    dat Micioni fratri adoptandum Aeschinum,\\
    sed Ctesiphonem retinet. hunc citharistriae\\
    lepore captum sub duro ac tristi patre\\
    frater celabat Aeschinus; famam rei,\\
    amorem in sese transferebat; denique\\
    fidicinam lenoni eripit. vitiaverat\\
    eidem Aeschinus civem Atticam pauperculam\\
    fidemque dederat hanc sibi uxorem fore.\\
    Demea iurgare, graviter ferre; mox tamen\\
    ut veritas patefactast, ducit Aeschinus\\
    vitiatam, potitur Ctesipho citharistriam.\\

}
\end{verse}

%\newpage

\section*{Paraphrasis}

Demea, duos habens filios, dat ex illis Aeschinum fratri Micioni in adoptionem, et Ctesiphonem alterum in sua potestate retinet: hunc,fidicinae festiva forma et ingenio irretitum, cum aspero et saevo patri subditus esset, frater Aeschinus occultabat; rumorem et amorem in se transferebat, simulans se esse amatorem citharistriae: denique eam lenoni per vim abripuit. Stupraverat quoque Aeschinus Pamphilam, virginem Atticam, pauperculam quidem, sed bonam, bonis civibus prognatam, pollicitusque fuerat se eam ducturum uxorem. Demea morosum et jurgiosum se praebebat, et iniquo animo ferebat. Sed cito, postquam id quod res erat manifestum fuit de psaltria, Aeschinum non sibi, sed Ctesiphoni fratri eam rapuisse, ducit Aeschinus uxorem Pamphilam a se compressam; Ctesipho fit compos fidicinae, placato et victo precibus Demea, patre severo et molesto.

\section*{Personae}

\begin{description}[noitemsep]
\item[(Prologus)]
\item[Micio] Senex
\item[Demea] Senex
\item[Sannio] Leno
\item[Aeschinus] Adulescens
\item[Bacchis] Meretrix
\item[Parmeno] Servos
\item[Syrus] Servos
\item[Ctesipho] Adulescens
\item[Sostrata] Matrona
\item[Canthara] Anus
\item[Geta] Servos
\item[Hegio] Senex
\item[Dromo] Puer
\item[Pamphila] Virgo
\item[(Cantor)]

\end{description}

\section*{AELII DONATI, V.\ C.\ ORATORIS URBIS ROMAE, PRAEFATIO IN ADELPHOS TERENTII.}

{\large
Haec Fabula palliata, ut ipsum indicat nomen, ex plurali numero, cum sit una: et masculino genere, cum sit Comoedia: ex Graeca Lingua, cum sit Latina, censetur. Potuit eam Terentius Fratres dicere: sed et Graeci nominis εὐφωνίαν perderet: et praeterea togata videretur; ad summam, non statim intelligeretur Menandri esse: quod Terentius in primis lectorem scire cupit: minus existumans laudis, proprias scribere, quam Graecas transferre. Est igitur Menandri: et a Fratrum facto, quibus argumentum nititur, nomen accipit. Hujus tota actio cum sit mixta ex utroque genere, ut fere Terentianae omnes, praeter Heautontimorumenon, tamen majori ex parte Motoria est: nam Statarios locos perpaucos habet. Prodest autem et delectat actu et stylo. In hac, primae partes sunt, ut quidam putant, Demeae: ut quidam, Syri. Quod si est, ut primas Syrus habeat; secundae Demeae erunt, tertiae Micionis, et sic deinceps. Quanquam etiam sunt qui putant, primas Micioni dandas, secundas Syro, tertias Demeae. Nam, quod ait Terentius, ``Senes qui primi venient'': non ad partes quas dicimus, sed ad ordinem pertinet exeuntium personarum. Haec etiam, ut cetera hujuscemodi Poemata, quinque Actus habeat necesse est, choris divisos a Graecis Poetis: quos etsi retinendi causa jam inconditos spectatores minime distinguunt Latini Comici, metuentes scilicet, ne quis fastidiosus finito Actu, velut admonitus abeundi, reliquae Comoediae fiat contemtor et surgat; tamen a doctis veteribus discreti atque disjuncti sunt, ut mox aperiemus post argumenti narrationem. In hac Prologus aliquanto lenior inducitur: qui magis etiam in se purgando, quam in adversariis laedendis est occupatus. Protasis est turbulenta, Epitasis clamosa, Catastrophe lenis. Quarum partium rationem diligentius in principio proposuimus, cum de Comoedia quaedam diceremus. Haec sane acta est ludis scenicis funebribus L.\ Aemilii Pauli: agentibus L.\ Ambivio et L.\ Turpione: qui cum suis gregibus etiam tum personati agebant. Modulata est autem tibiis dextris, id est, Lydiis, ob seriam gravitatem, qua fere in omnibus Comoediis utitur hic Poeta. Saepe tamen, mutatis per scenam modis, cantica mutavit: quod significat titulus scenae, habens subjectas personis literas, M.\ M.\ C. Item Diverbia ab histrionibus crebro pronuntiata sunt, quae significantur D.\ et M.\ literis secundum personarum nomina praescriptis, in eo loco ubi incipit scena. Annotandum sane, quod haec Fabula προτατικὸν πρόσωπον non habet, hoc est, personam, quae ad argumentum nihil attineat, quaeque sit assumta extrinsecus, ut est in Andria Sosia. Hanc dicunt ex Terentianis secundo loco actam, etiam tum rudi nomine Poetae. Itaque sic pronuntiatam, Adelphoe Terenti, non Terenti Adelphoe: quod adhuc magis de Fabulae nomine Poeta, quam de Poetae nomine Fabula commendabatur. In hac quidem spectatur quid intersit inter rusticam et urbanam vitam: mitem et asperam: coelibis et mariti: veri patris et per adoptionem facti. Quibus propositis ad exemplum, vitanda perinde, fugiendaque, Terentius monstrans, artificis Poetae per totam Fabulam obtinet laudem.\\

}

\section*{AELII DONATI ARGUMENTUM IN ADELPHOS.}

Ex duobus Atticis fratribus, alter quidem, Demea nomine, rus coluit, uxorem duxit, filios suscepit duos, Aeschinum et Ctesiphonem. At alter, Micio nomine, uxorem non duxit, et filios procreare noluit : sed sibi filium fratris Aeschinum adoptavit: atque ita indulgenter eduxit a parvulo, ut effuse luxuriatus adolescens ad postremum civem Atticam virginem vitiaret, captus amore ejus: quo facto, et cum matre puellae pepigit nuptias ejusdem, quam vitiavit. Cumque rem gestam ad patris, a quo adoptatus fuerat, conscientiam jam jamque perlaturus esset, precibus Ctesiphonis, fratris sui, qui apud durum patrem atque agrestem Demeam parcius atque arctius haberetur, impulsus est, et ut eidem a lenone raperet meretricem: quo facto, multiplici errore completur Fabula. Nam et Demea cum hoc ipso, id est, cum Micione litigabat, tanquam cum eo qui corruperit adolescentem adoptatum, in mores perditos, nesciens suum sibi filum Ctesiphonem esse corruptum, eluditurque a Syro et Micione per totam Fabulam. Et mater puellae jam decimo mense post raptum virginis Aeschinum credit sibi ipsi rapuisse meretricem. Quae perturbatio cito in tranquillum redacta est. Nam, re comperta de vitio virginis, Micio dat civem Aeschino quam concnpiverat: ejusque matrem ipse accipit. Deprehenso vero Ctesiphone in amore meretricis, primo irascitur Demea: post lenitur, atque habendae ejus meretricis licentiam praebet. 

Primus actus haec continet: Micionis solius primo verba: et post, ejusdem et Demeae jurgium. 

Secundus actus haec continet: lenonis alteram rixam adversus Aeschinum pro puella: ejusdem apud Syrum querelas: laetitiam Ctesiphonis: obsessionem amicae: et ejusdem gratiarum actiones apud Aeschinum. 

Tertius actus haec continet: trepidationem matris Sostratae et Cantharae nutricis ob parturientem Pamphilam vitiatam ab Aeschino: Getam nuntiantem dominae suae per errorem, quod sibi rapuerit Aeschinus meretricem: reditum in scenam Demeae: ejusdemque cum Syro ludificante sermocinationem: interventum Hegionis, cum querela apud eundem Demeam de facto Aeschini, de  consolatione Sostratae. 

Quartus actus haec continet: Ctesiphonis cum Syro colloquium: delusionem Demeae: ejusdemque in scenam interventum, atque secundam frustrationem per Syrum factam: Micionis cum Hegione sermonem: querelam Aeschini de rebus suis, ejusdemque cum Micione patre facetissimam dissertationem: Demeae reditum in scenam ex errore, in quem eum conjecerat Syrus: et renovatum cum fratre ejusdem jurgium: processionem in scenam temulenti Syri.

Quintus actus haec continet: deprehensionem Ctesiphonis cum meretrice: tertium cum Micione jurgium Demeae, ejusdemque vitae pristinae correptionem: et per eum multa in Comoedia Nova, hoc est, blandimentum circa Aeschinum, et affabilitatem circa Getam: conciliationem Syri et uxoris ejus: et veniam circa Ctesiphonem: permissionemque habendae meretricis. 

Servatur autem per totam Fabulam mitis Micio, saevus Demea, leno avarus, callidus Syrus, timidus Ctesipho, liberalis Aeschinus, pavidae mulieres, gravis Hegio. 

In dividendis actibus Fabulae identidem meminerimus; primo paginarum dinumerationem neque Graecos, neque Latinos servasse: cum ejus distributio hujusmodi rationem habeat, ut, ubi attentior spectator esse potuerit, longior actus sit: ubi fastidiosior, brevior atque contractior. Deinde etiam illud, in eundem actum posse conjici, et tres et quatuor scenas introeuntium atque exeuntium personarum. 

Facta autem haec una est de duabus Fabulis, Adelphis Menandri, et Commorientibus Diphili.

\poemtitle*{Prologus}

\settowidth{\versewidth}{in agendo partem ostendent. facite aequanimitas}
\begin{verse}[\versewidth]\poemlines{5}
{\large
    Postquam poeta sensit scripturam suam\\
    ab iniquis observari, et advorsarios\\
    rapere in peiorem partem quam acturi sumus,\\
    indicio de se ipse erĭt, vos eritis iudices\\
    laudin an vitio duci factum oporteat.\\
    Synapothnescontes Diphili comoediast:\\
    eam Commorientis Plautu' fecit fabulam.\\
    in Graeca adulescens est qui lenoni eripit\\
    meretricem in prima fabula: eum Plautus locum\\
    reliquit integrum, eum hic locum sumpsit sibi\\
    in Adelphos, verbum de verbo expressum extulit.\\
    eam nos acturi sumu' novam: pernoscite\\
    furtumne factum existumetis an locum\\
    reprehensum qui praeteritu' neglegentiast.\\
    nam quod ĭsti dicunt malevoli, homines nobilis\\
    hunc adiutare adsidueque una scribere,\\
    quod ĭlli maledictum vehemens esse existumant,\\
    eam laudem hic ducit maxumam quom illis placet\\
    qui vobis univorsis et populo placent,\\
    quorum opera in bello in otio in negotio\\
    suo quisque tempore usust sine superbia.\\
    de(h)inc ne exspectetis argumentum fabulae,\\
    senes qui primi venient ĭ partem aperient,\\
    in agendo partem ostendent. facite aequanimitas\\
    poetae ad scribendum augeat industriam.\\
}
   
\end{verse}

\section*{Paraphrasis}

Postquam Poeta Terentius intellexit Comoedias, quas scripsit, curiose explorari, et notari ab improbis hominibus atque adversariis obtrectandi gratia, et Comoediam, quam exhibituri sumus, ab his reprehendi et culpari; ipsemet quid fecerit vobis indicabit; vos judicabitis utrum factum laude dignum sit, an vituperio. Diphilus Poeta comicus Graecus Comoediam fecit, quam Graece Synapothnescontes inscripsit: eam Plautus Latine composuit, et Commorientes nominavit. In Graeca Diphili quidam inducitur adolescens rapiens lenoni meretricem, in prima parte Fabulae; eum locum reliquit Plautus intactum: hic vero Poeta Terentius hunc locum transtulit in Adelphos, et ex Graeco in Latinum vertens ornatiorem reddidit. Eam Comoediam novam, nempe Adelphos, vobis, spectatores, nunc acturi et recitaturi sumus. Nunc perspicite an censere debeatis, hunc locum, quem mutuatus est Terentius, furti vitio accusandum sit, an potius dicendum ab eo resumtum et retentum esse, postquam Plautus eum pro derelicto negligenter habuerat. Nam, quod adversarii isti in vulgus jactant, homines nobili et patricia stirpe clarissimos et operam suam dare, et perpetuo una Comoedias facere; quod illi vituperium grave arbitrantur, id sibi ingenti gloriae tribuit, cum illis gratus et acceptus est, qui vobis, quotquot hic estis, et universo populo grati et accepti sunt: quorum beneficia, in militia et in pace, commode in omnes collata fuerunt sine fastu et ostentatione. Exinde nolite expectare ut argumentum fabulae vobis enarretur. Senes, qui primi in proscenium prodibunt, ii partem explicabunt ipso initio, partem in reliquo dramate. Concedite, ut aequo et propitio animo adsitis, quo labor et industria Poetae ad faciendas Comoedias magis ac magis excitetur.
