%\section*{O autoru}



\section*{De mortis metu}

Hortatur amicum ut perseveret in philosophia et in emendando animo: quo majores enim profectus faciat, eo majorem voluptatem sensurum esse, virili animo assumpto et puerilitate deposita. Praecipua autem utilitas hujus studii est contemptus animae, sive mortis: quam quum ex frivolis causis contemnant interdum homines, e solidioribus virtutis causis multo magis eam esse parvi faciendam monet. Vitam itque una cum bonis eam comitantibus non esse amandam, sed omnem pro illa sollicitudinem deponendam, quum ex tot levissimis causis mors nos occupare possit. Finem epistolae facit dictum Epicuri de veris divitiis. Hoc emblema per se egregium satis bene cum antecedentibus conjungi potest, quandoquidem ei, qui paucis contentus, nec avidus est rerum in vita supervacanearum, minus metus objicere potest mora, quam iis, qui contrariam vitae rationem inierunt: quanquam in hisce clausulis illam arctam cum antecedentibus conjunctionem quaerere velle ipsius auctoris consilio repugnaret.

\newpage

\section*{Pročitajte naglas latinski tekst.}

%Naslov prema izdanju

Sen. ep. 4

\medskip

{\large
\noindent Seneca Lucilio suo salutem

\medskip


\noindent Persevera ut coepisti et quantum potes propera, quo diutius frui emendato animo et conposito possis. Frueris quidem etiam dum emendas, etiam dum  conponis; alia tamen illa voluptas est, quae percipitur ex contemplatione mentis ab omni labe purae et splendidae.

Tenes utique memoria, quantum senseris gaudium, cum praetexta posita sumpsisti virilem togam et in forum deductus es; maius expecta, cum puerilem animum deposueris et te in viros philosophia transscripserit. Adhuc enim non pueritia sed, quod est gravius, puerilitas remanet. Et hoc quidem peior est, quod auctoritatem habemus senum, vitia puerorum, nec puerorum tantum sed infantum. Illi levia, hi falsa formidant, nos utraque.

Profice modo; intelleges quaedam ideo minus timenda, quia multum metus adferunt. Nullum malum est magnum, quod extremum est. Mors ad te venit; timenda erat, si tecum esse posset; sed necesse est aut non perveniat aut transeat.

``Difficile est,'' inquis, ``animum perducere ad contemptionem animae.'' Non vides, quam ex frivolis causis contemnatur? Alius ante amicae fores laqueo pependit, alius se praecipitavit e tecto, ne dominum stomachantem diutius audiret, alius, ne reduceretur e fuga, ferrum adegit in viscera. Non putas virtutem hoc effecturam, quod efficit nimia formido? Nulli potest secura vita contingere, qui de producenda nimis cogitat, qui inter magna bona multos consules numerat.

Hoc cotidie meditare, ut possis aequo animo vitam relinquere, quam multi sic conplectuntur et tenent, quomodo qui aqua torrente rapiuntur spinas et aspera.

Plerique inter mortis metum et vitae tormenta miseri fluctuantur et vivere nolunt, mori nesciunt.

Fac itaque tibi iucundam vitam omnem pro illa sollicitudinem deponendo. Nullum bonum adiuvat habentem, nisi ad cuius amissionem praeparatus est animus; nullius autem rei facilior amissio est, quam quae desiderari amissa non potest. Ergo adversus haec, quae incidere possunt etiam potentissimis, adhortare te et indura.

De Pompei capite pupillus et spado tulere sententiam, de Crasso crudelis et insolens Parthus; Gaius Caesar iussit Lepidum Dextro tribuno praebere cervicem, ipse Chaereae praestitit. Neminem eo fortuna provexit, ut non tantum illi minaretur, quantum permiserat. Noli huic tranquillitati confidere; momento mare evertitur. Eodem die ubi luserunt navigia, sorbentur.

Cogita posse et latronem et hostem admovere iugulo tuo gladium: Ut potestas maior absit, nemo non servus habet in te vitae necisque arbitrium. Ita dico: quisquis vitam suam contempsit, tuae dominus est. Recognosce exempla eorum, qui domesticis insidiis perierunt, aut aperta vi aut dolo; intelleges non pauciores servorum ira cecidisse quam regum. Quid ad te itaque, quam potens sit quem times, cum id, propter quod times, nemo non possit?

At si forte in manus hostium incideris, victor te duci iubebit; eo nempe, quo duceris. Quid te ipse decipis et hoc nunc primum, quod olim patiebaris, intellegis? Ita dico: ex quo natus es, duceris. Haec et eiusmodi versanda in animo sunt, si volumus ultimam illam horam placidi expectare, cuius metus omnes alias inquietas facit. 

Sed ut finem epistulae inponam, accipe, quod mihi hodierno die placuit. Et hoc quoque ex alienis hortulis sumptum est. ``Magnae divitiae sunt lege naturae composita paupertas.'' Lex autem illa naturae scis quos nobis terminos statuat? Non esurire, non sitire, non algere. Ut famem sitimque depellas, non est necesse superbis adsidere liminibus nec supercilium grave et contumeliosam etiam humanitatem pati, non est necesse maria temptare nec sequi castra; parabile est, quod natura desiderat, et adpositum.

Ad supervacua sudatur. Illa sunt, quae togam conterunt, quae nos senescere sub tentorio cogunt, quae in aliena litora inpingunt. Ad manum est, quod sat est. Cui cum paupertate bene convenit, dives est. VALE.


}


%\section*{Analiza}

%1

%{\large
%\noindent Ita fac, mi Lucili; \\
