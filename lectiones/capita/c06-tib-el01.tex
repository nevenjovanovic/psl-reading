%\section*{O autoru}

Haec prima Elegia, quae tempore postrema fuit, praefixa ceteris videtur ut constaret quodnam vitae genus Tibullus expeteret. Lapsus enim opibus exhaustoque opimo matrimonio, quod amoribus videtur profudisse, secessit in agellum. Unde Horat.\ ``Non tu corpus eras sine pectore: Di tibi formam, Di tibi divitias dederant artemque fruendi.'' 

Haec igitur noster scripsit Tibullus in praediolo suo Pedano, ad Messalam Corvinum in Ciliciam proficiscentem; profiteturque malle se pauperem in otio degere vitam quam divitias bellicis laboribus comparare.

%\poemtitle*{Elegia I.}

%Naslov prema izdanju




\settowidth{\versewidth}{Nunc levis est tractanda Venus, dum frangere postes}
\begin{verse}[\versewidth]\poemlines{5}
\begin{altverse}
{\large
Divitias alius fulvo sibi congerat auro\\
Et teneat culti iugera multa soli,\\
Quem labor adsiduus vicino terreat hoste,\\
Martia cui somnos classica pulsa fugent:\\
Me mea paupertas vita traducat inerti,\\
Dum meus adsiduo luceat igne focus.\\
Ipse seram teneras maturo tempore vites\\
Rusticus et facili grandia poma manu;\\
Nec spes destituat, sed frugum semper acervos\\
Praebeat et pleno pinguia musta lacu.\\
Nam veneror, seu stipes habet desertus in agris\\
Seu vetus in trivio florida serta lapis,\\
Et quodcumque mihi pomum novus educat annus,\\
Libatum agricolae ponitur ante deo.\\
Flava Ceres, tibi sit nostro de rure corona\\
Spicea, quae templi pendeat ante fores,\\
Pomosisque ruber custos ponatur in hortis,\\
Terreat ut saeva falce Priapus aves.\\
Vos quoque, felicis quondam, nunc pauperis agri\\
Custodes, fertis munera vestra, Lares.\\
Tunc vitula innumeros lustrabat caesa iuvencos,\\
Nunc agna exigui est hostia parva soli.\\
Agna cadet vobis, quam circum rustica pubes\\
Clamet ``io messes et bona vina date''.\\
Iam modo iam possim contentus vivere parvo\\
Nec semper longae deditus esse viae,\\
Sed Canis aestivos ortus vitare sub umbra\\
Arboris ad rivos praetereuntis aquae.\\
Nec tamen interdum pudeat tenuisse bidentem\\
Aut stimulo tardos increpuisse boves,\\
Non agnamve sinu pigeat fetumve capellae\\
Desertum oblita matre referre domum.\\
At vos exiguo pecori, furesque lupique,\\
Parcite: de magno est praeda petenda grege.\\
Hic ego pastoremque meum lustrare quotannis\\
Et placidam soleo spargere lacte Palem.\\
Adsitis, divi, neu vos e paupere mensa\\
Dona nec e puris spernite fictilibus.\\
Fictilia antiquus primum sibi fecit agrestis\\
Pocula, de facili conposuitque luto.\\
Non ego divitias patrum fructusque requiro,\\
Quos tulit antiquo condita messis avo:\\
Parva seges satis est, satis requiescere lecto\\
Si licet et solito membra levare toro.\\
Quam iuvat inmites ventos audire cubantem\\*
Et dominam tenero continuisse sinu\\
%\newpage
Aut, gelidas hibernus aquas cum fuderit Auster,\\*
Securum somnos igne iuvante sequi.\\
Hoc mihi contingat. Sit dives iure, furorem\\*
Qui maris et tristes ferre potest pluvias.\\
O quantum est auri pereat potiusque smaragdi,\\
Quam fleat ob nostras ulla puella vias.\\
Te bellare decet terra, Messalla, marique,\\
Ut domus hostiles praeferat exuvias;\\
Me retinent vinctum formosae vincla puellae,\\
Et sedeo duras ianitor ante fores.\\
Non ego laudari curo, mea Delia; tecum\\
Dum modo sim, quaeso segnis inersque vocer.\\
Te spectem, suprema mihi cum venerit hora,\\
Te teneam moriens deficiente manu.\\
Flebis et arsuro positum me, Delia, lecto,\\
Tristibus et lacrimis oscula mixta dabis.\\
Flebis: non tua sunt duro praecordia ferro\\
Vincta, neque in tenero stat tibi corde silex.\\
Illo non iuvenis poterit de funere quisquam\\
Lumina, non virgo, sicca referre domum.\\
Tu manes ne laede meos, sed parce solutis\\
Crinibus et teneris, Delia, parce genis.\\
Interea, dum fata sinunt, iungamus amores:\\
Iam veniet tenebris Mors adoperta caput,\\
Iam subrepet iners aetas, nec amare decebit,\\
Dicere nec cano blanditias capite.\\
Nunc levis est tractanda Venus, dum frangere postes\\
Non pudet et rixas inseruisse iuvat.\\
Hic ego dux milesque bonus: vos, signa tubaeque,\\
Ite procul, cupidis volnera ferte viris,\\
Ferte et opes: ego conposito securus acervo\\
Despiciam dites despiciamque famem. \\

}
\end{altverse}
\end{verse}

\newpage

\section*{Textus cum paraphrasi}

{\large

Divitias alius fulvo sibi congerat auro et teneat culti iugera multa soli, quem labor adsiduus vicino terreat hoste, Martia cui somnos classica pulsa fugent: me mea paupertas vita traducat inerti, dum meus adsiduo luceat igne focus.\\

}

\noindent Alius sibi opes accumulet auro fulvo, multaque praedia terrae aratae possideat: quem jugis anxietas propinquo hoste perterrefaciat, cui tuba bellica pulsa toporem excutiunt. Egestas mea me vita ignava transmittat, modo caminus meus perenni flamma fulgeat.\\


{\large

\noindent Ipse seram teneras maturo tempore vites rusticus et facili grandia poma manu; nec spes destituat, sed frugum semper acervos praebeat et pleno pinguia musta lacu.\\

}

\noindent Idem novellas vites agricola plantabo tempore opportuno, et pro ceras arbores mala ferentes industria manu. Nec me fallat expectatio, at perpetuo frumentorum cumulos suppeditet, et vina suavia e lacu referto.\\

{\large

\noindent Nam veneror, seu stipes habet desertus in agris seu vetus in trivio florida serta lapis, et quodcumque mihi pomum novus educat annus, libatum agricolae ponitur ante deo.\\

}

\noindent Colo etenim, sive truncus in agris relictus, sive saxum antiquum in trivia gerit corollas floridas. Et quosvis fructus gignit mihi novus annus, prius decerpti Deo Ruricolae offeruntur. \\


{\large
\noindent Flava Ceres, tibi sit nostro de rure corona spicea, quae templi pendeat ante fores, pomosisque ruber custos ponatur in hortis, terreat ut saeva falce Priapus aves.\\

}


\noindent О Ceres flava, corona spicea tibi sit ex agris meis, quae ad aedis januam appendatur; et tutor rubicundus Priapus statuatur in hortis pomiferis, ut volucres minaci falce territet.\\ 

{\large
\noindent Vos quoque, felicis quondam, nunc pauperis agri custodes, fertis munera vestra, Lares.\\

}

\noindent Vos etiam, o Lares tutores praedii mei quondam prosperi, jam tenuis, dona vestra accipitis. \\

{\large
\noindent Tunc vitula innumeros lustrabat caesa iuvencos, nunc agna exigui est hostia parva soli.\\
}

\noindent Tum bucula immolata plurimos boves expiabat; jam vero agna ampla est victima agri modici.\\ 

{\large
\noindent Agna cadet vobis, quam circum rustica pubes clamet ``io messes et bona vina date''.\\

}

\noindent Agna vobis mactabitur, circa quam agrestis juventus vociferetur: Io tribuite segetes, et vina dulcia. \\

{\large
\noindent Iam modo iam possim contentus vivere parvo nec semper longae deditus esse viae, sed Canis aestivos ortus vitare sub umbra arboris ad rivos praetereuntis aquae.\\

}

\noindent Ego namque nunc modico contentus vitam degere queo, neque perpetuo obnoxius esse longo itineri; at ortum aestivum Caniculae devitare sub umbra arboris, ad rivos aquae praeterfluentis. \\

{\large
\noindent Nec tamen interdum pudeat tenuisse bidentem aut stimulo tardos increpuisse boves, non agnamve sinu pigeat fetumve capellae desertum oblita matre referre domum.\\

}


\noindent Neque tamen aliquando me pigeat pastinum traxisse aut boves lentos aculeo instigasse. Nec me pudeat gremio in aedes reportare aut agnam, aut caprae partum a matre immemori derelictum.\\ 

\newpage

{\large
\noindent At vos exiguo pecori, furesque lupique, parcite: de magno est praeda petenda grege.\\

}

\noindent Vos autem, o fures et lupi, abstinete ab exiguo pecore : praeda vobis est rapienda ex magno grege. \\

{\large
\noindent Hic ego pastoremque meum lustrare quotannis et placidam soleo spargere lacte Palem.\\

}

\noindent Tum singulis annis consuevi, et pastorem meum expiare, et mitem Palem lacte perfundere. \\

{\large
\noindent Adsitis, divi, neu vos e paupere mensa dona nec e puris spernite fictilibus.\\

}

\noindent O Di, praesto sitis, neque vos despicite munera e tenui mensa, neque e levibus samiis seu fictilibus.\\

{\large
\noindent Fictilia antiquus primum sibi fecit agrestis pocula, de facili conposuitque luto.\\

}

\noindent Priscus rusticus primo sibi finxit vasa samia, seu fictilia, et e levi argilla conflavit. \\

{\large
\noindent Non ego divitias patrum fructusque requiro, quos tulit antiquo condita messis avo: parva seges satis est, satis requiescere lecto si licet et solito membra levare toro.\\

}

\noindent Non ego majorum opes et reditus desidero, quos seges reposita attulit priscis avis. Messis exigua sufficit; sufficit quiescere cubili, et unico lectulo artus relaxare. \\

\newpage

{\large
\noindent Quam iuvat inmites ventos audire cubantem et dominam tenero continuisse sinu aut, gelidas hibernus aquas cum fuderit Auster, securum somnos igne iuvante sequi.\\

}

\noindent Quam me delectat requiescentem saeva flamina auscultare, blandisque amantem premere complexibus. Vel, quando Notus hyemalis undas immiserit, tranquillum consectari quietem pluvia conciliante. \\


{\large
\noindent Hoc mihi contingat. Sit dives iure, furorem qui maris et tristes ferre potest pluvias.\\

}

\noindent Id mihi accidat. Is opulentus merito sit, qui pati potest iras pelagi et graves imbres. \\

{\large
\noindent O quantum est auri pereat potiusque smaragdi, quam fleat ob nostras ulla puella vias.\\

}

\noindent О potius quicquid est auri et smaragdi intereat, quam ut puella aliqua ploret propter itinera nostra. \\

{\large
\noindent Te bellare decet terra, Messalla, marique, ut domus hostiles praeferat exuvias; me retinent vinctum formosae vincla puellae, et sedeo duras ianitor ante fores.\\

}

\noindent О Messalla par est, te certare terra et mari, ut aedes tuae praetendant hostica spolia. Me catenae pulchrae puellae ligatum coercent, et ad januam saevam quasi ostiarius resideo. \\

{\large
\noindent Non ego laudari curo, mea Delia; tecum dum modo sim, quaeso segnis inersque vocer.\\

}

\noindent Non equidem commendari expeto, о Delia mea, modo tecum verser, appeller, oro, ignavus et otiosus. \\

\newpage

{\large
\noindent Te spectem, suprema mihi cum venerit hora, te teneam moriens deficiente manu.\\

}

\noindent Ego intuear te, ubi mihi ultima hora advenerit, te languente dextera mortem obiens teneam. \\

{\large
\noindent Flebis et arsuro positum me, Delia, lecto, tristibus et lacrimis oscula mixta dabis. Flebis: non tua sunt duro praecordia ferro vincta, neque in tenero stat tibi corde silex. Illo non iuvenis poterit de funere quisquam lumina, non virgo, sicca referre domum.\\

}

\noindent О Delia, me etiam lugebis collocatum in toro flagraturo, et basia praebebis moestis fletibus confusa; plorabis, viscera tua non sunt ferro aspero constricta, nec lapis stat tibi in molli pectore : nullus juvenis, nec puella, domum referre poterit aridos oculos ex illis exequiis. \\

{\large
\noindent Tu manes ne laede meos, sed parce solutis crinibus et teneris, Delia, parce genis.\\

}


\noindent Tu ne viola umbram meam : verum passis comis abstine, о Delia, et ori delicato tempera. \\

{\large
\noindent Interea, dum fata sinunt, iungamus amores: iam veniet tenebris Mors adoperta caput, iam subrepet iners aetas, nec amare decebit, dicere nec cano blanditias capite.\\

}

\noindent Interim donec sors patitur, conciliemus amores : mox letum adveniet caput caligine obtectum : mox illabetur ignava aetas, nec amare conveniet, nec delicias dicere cano vertice. \\

{\large
\noindent Nunc levis est tractanda Venus, dum frangere postes non pudet et rixas inseruisse iuvat.\\

}

\noindent Jam expeditus amor exercendus est, quamdiu minime piget fores rumpere, et delectat jurgia immiscuisse. \\

{\large
\noindent Hic ego dux milesque bonus: vos, signa tubaeque, ite procul, cupidis volnera ferte viris, ferte et opes: ego conposito securus acervo despiciam dites despiciamque famem. \\

}

\noindent Нас in re ego ductor et miles fortis; vos, vexilla et buccinae, longe facessite. Avidis hominibus plagas immittite. Divitias etiam adducite. Ego congesto cumulo tranquillus locupietes aspernabor, et contemnam esuriem.

%1

