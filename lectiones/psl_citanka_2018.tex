\documentclass[a4paper,12pt,twoside]{report}
\usepackage[latin]{babel}

\usepackage{fontspec}
\usepackage{verse}
\defaultfontfeatures{Ligatures=TeX}

\usepackage{import}
\usepackage[small,sf,bf]{titlesec}
\usepackage{tabto}
\usepackage{ulem}
\usepackage{hyperref}
\usepackage{enumitem}

\usepackage{fancyhdr}
\renewcommand{\chaptermark}[1]{\markboth{#1}{}}
\renewcommand{\sectionmark}[1]{\markright{#1}}
\pagestyle{fancy}
\fancyhf{}
\fancyhead[LE,RO]{\thepage}
\fancyhead[RE]{\itshape\nouppercase{Prevođenje s latinskog – čitanka}}
\fancyhead[LO]{\itshape\nouppercase{\leftmark}}
\renewcommand{\headrulewidth}{0pt}

\usepackage{titling}
\newcommand{\subtitle}[1]{%
  \posttitle{%
    \par\end{center}
    \begin{center}\large#1\end{center}
    \vskip0.5em}%
}
 
\setmainfont{Old Standard TT}
\setsansfont{Old Standard TT}
%\setsansfont{Tahoma}

\hyphenation{δυσ-σέ-βει-αν βού-λεσ-θαί κα-τη-γο-ρού-σης τα-χέ-ως πε-πλημ-μέ-λη-κε νε-α-νί-ας αὐ-τῇ ἀ-φε-λό-με-νος ἀ-πή-γε-το ἐ-πέ-πληξ-άς ἠ-γό-μην ἤ-νεγ-κεν νο-μί-ζει ἄν-θρω-πον παν-το-δα-πά Αἰ-θί-οψ-ιν ἀν-έρ-χε-ται φρον-τί-δων αὐ-τὸς δι-ῃ-ρη-μέ-νος κλέπ-τον-τας ἑ-κα-τόμ-βῃ μέ-γισ-τον κιν-δύ-νου}

\hyphenation{Ελ-λή-νων ξυν-έ-μει-νεν ἐ-πι-όντ-ων ἀνα-σκευ-α-σά-με-νοι ἀ-πω-σά-με-νοι Λα-κε-δαι-μό-νι-οι Ελ-λη-νες δι-ε-φά-νη}

\hyphenation{εὐ-δο-κι-μή-σας ἐν-ταῦ-θα βα-σι-λεύ-εις ἐ-χρή-σα-το συλ-ληφ-θέν-τος}

\hyphenation{θε-ρά-πον-τες ἡ-γοῦ-μαι}

\hyphenation{ἐπι-φα-νέσ-τε-ρον το-σοῦ-τον ἔ-χον-τας συγ-γιγ-νο-μέ-νους μᾶλ-λον με-γίσ-την}

\hyphenation{ἐπ-έσ-κηπ-τε}

\hyphenation{κοι-νω-νοῦ-σιν δη-λῶ-σαι δι-α-λε-γο-μέ-νους πλη-σι-ά-ζον-τας πα-ρα-λι-πεῖν}

\hyphenation{καρ-πῶν ὀ-νο-μα-ζό-με-νον τηκ-τὰ ὅ-σα}

\hyphenation{με-μά-θη-κας Μά-λισ-τα πο-λυ-μα-θής}

\hyphenation{δι-α-βε-βλη-μέ-νος Α-ρι-στό-βου-λος Σω-κρά-τους Δı-α-τ
ρι-βαί ἐξ-ελ-κύ-σαι Ἐγ-χει-ρί-δι-ον}

\hyphenation{δά-μα-λιν δύ-να-μαι}

\hyphenation{πυ-θο-μέ-νου ἀ-πο-κρί-νε-ται τρα-πέ-ζαις ἀ-πέσ-τει-λαν γρά-ψαν-τος με-τα-τί-θη-σι με-γά-λου Τισ-σα-φέρ-νῃ προσ-ε-πι-μετ-ρῆ-σαι Α-θη-ναί-οις Τισ-σα-φέρ-νην Αλ-κι-βι-ά-δην πα-τρί-δα}

\hyphenation{ἡ-γοῦν-ται ἀ-πο-ρω-τά-των πό-λεις Ras-pra-va Ἀ-πο-λο-γί-α Λα-κε-δαı-μο-νί-ων Κυ-νη-γε-τι-κός}

\hyphenation{πλε-όν-των γε-ωρ-γὸς ἕ-τε-ρον}

\hyphenation{κα-θεῖ-ναι προ-ελ-θοῦ-σαν ἀ-πο-δοῦ-ναι ἀ-πο-φαί-νον-τος ἀ-πεσ-τά-λη ἅ-παν-τας τρί-πο-δα τρί-πο-δος κα-θι-ερώ-θη}

\hyphenation{χρεί-αν παν-τὸς ἀν-επί-σκεπ-τον ἀ-πο-θα-νόν-των πα-ρα-σκευ-ά-ζον-τας φρον-τί-ζον-τας πολ-λοὺς κτω-μέ-νους ἐ-λατ-τοῦσ-θαι ἀ-με-λοῦν-τας ἀ-θε-ρά-πευ-τον κτη-μά-των πει-ρω-μέ-νους ὀ-λι-γω-ροῦν-τας δε-ο-μέ-νων ἐ-ῶν-τας ἔ-μοι-γε}

\hyphenation{αὐ-το-κρά-το-ρας Λά-μα-χον ποι-ή-σαν-τες ἑξή-κον-τα Σε-λι-νουν-τί-ους χρη-μά-των μισ-θόν Νι-κη-ρά-του πε-ρι-γίγ-νη-ται}



\begin{document}

\title{Prevođenje s latinskog}
\subtitle{Čitanka}
\author{Odsjek za klasičnu filologiju\\
Filozofski fakultet Sveučilišta u Zagrebu}
\maketitle

\clearpage
\thispagestyle{empty}


%\frontmatter


\chapter*{Predgovor}

\section*{O ovoj čitanci}

Izbor koji je pred vama donosi dijelove četiri djela četvorice uglednih autora rimske književnosti: Terencija, Horacija, Tibula i Seneke. Njihova su djela (\textit{Braća, Prva knjiga oda, Prva knjiga elegija, Prva knjiga moralnih pisama Luciliju}) propisana lektira kolegija \textit{Prevođenje s latinskog}, obaveznog na trećoj godini preddiplomskog studija latinskoga jezika i književnosti na Filozofskom fakultetu Sveučilišta u Zagrebu.

Zadatak je studenata da, uz pomoć uobičajene referentne literature, kod kuće prirede svaku cjelinu, tako da na nastavi budu spremni latinski tekst pročitati, prevesti na hrvatski, i interpretirati ga pod vodstvom nastavnika.

Valja upozoriti da ova čitanka nipošto ne sadrži \textit{cjelokupnu} propisanu lektiru, koja je zapravo bitno opsežnija. Petnaest jedinica ovog izbora namijenjeno je zajedničkom prevođenju, komentiranju i vježbanju na seminaru. Ostatak propisanih štiva svaki student mora, da bi uspješno položio ispit, prirediti sam; samostalno čitanje i samostalan rad na tekstu osnova su latinske jezične kompetencije.

Osim samih izvornih tekstova, čitanka donosi i kratke uvode, primjere raščlanjivanja, i (gdje je bilo moguće ili umjesno) latinske parafraze, preuzete iz devetnaestostoljetnih izdanja, prvenstveno iz niza koji je \textit{in usum Delphini} (tzv.\ \textit{Delphin Classics)} 1819-30.\ objavljivao londonski izdavač Abraham John Valpy. Digitalni faksimili svih tih knjiga danas su dostupni putem interneta, ponajviše u zbirci \textit{Internet Archive}. Bibliografske podatke donosimo u nastavku. 

Školska izdanja s latinskim komentarima i parafrazama smatramo iznimno korisnim za stjecanje najvažnije vještine na studiju latinskog – za tečno čitanje i razumijevanje latinskog teksta.

Ovaj je izbor priredio Neven Jovanović, nastavnik Odsjeka za klasičnu filologiju Filozofskog fakulteta Sveučilišta u Zagrebu.

%\newpage


\medskip

U Zagrebu, rujna 2018.

\section*{Bibliografija}

{
\setlength{\parindent}{0pt}

Albii Tibulli opera omnia ex editione I.\ G.\ Huschkii, in usum Delphini. Vol.\ 1. Londini: Valpy, 1822. \href{https://archive.org/details/delphinclassics173valp}{Internet}.

L.\ Annaei Senecae pars prima, sive opera philosophica, rec.\ M.\ N.\ Bouillet, vol.\ 3. Parisiis: Lemaire, 1828. \href{https://archive.org/details/lannsenecparspr03bouigoog}{Internet}.

Pub. Terentii Afri comoediae sex, ex editione Westerhoviana, in usum Delphini. Vol.\ 2. Londini: Valpy, 1824. \href{https://archive.org/details/pubterentiiafric02tereuoft}{Internet}.

Quinti Horatii Flacci opera omnia, ex editione J.\ C.\ Zeunii, in usum Delphini. Vol.\ 1.  Londini: Valpy, 1825. \href{https://archive.org/details/delphinclassics57valp}{Internet}.

}

%\mainmatter

\part{L.\ Annaei Senecae ad Lucilium epistulae morales}

\chapter{Epistula I.}

\import{capita/}{c01-sen-ep01.tex}

\chapter{Epistula II.}

\import{capita/}{c02-sen-ep02.tex}

\chapter{Epistula III.}

\import{capita/}{c03-sen-ep03.tex}

\chapter{Epistula IV.}

\import{capita/}{c04-sen-ep04.tex}

\chapter{Epistula V.}

\import{capita/}{c05-sen-ep05.tex}

\part{Albii Tibulli elegiarum liber primus}

\chapter{Elegia I.}


\import{capita/}{c06-tib-el01.tex}

\chapter{Elegia II.}


\import{capita/}{c07-tib-el02.tex}

\chapter{Elegia III.}


\import{capita/}{c08-tib-el03.tex}




\part{Terentii Adelphoe}

\chapter{Adelphoe, didascalia et prologus.}


\import{capita/}{c09-ter-adelph00.tex}

\chapter{Adelphoe, actus I, scaena I.}


\import{capita/}{c10-ter-adelph01.tex}

\chapter{Adelphoe, actus I, scaena II.}


\import{capita/}{c11-ter-adelph02.tex}

\chapter{Adelphoe, actus II, scaena I.}


\import{capita/}{c12-ter-adelph03.tex}


\part{Horatii odarum liber primus}

\chapter{Ode I.}


\import{capita/}{c13-hor-od1-1.tex}

\chapter{Ode II.}


\import{capita/}{c14-hor-od1-2.tex}

\chapter{Ode III.}


\import{capita/}{c15-hor-od1-3.tex}

%\clearpage
%\thispagestyle{empty}
% kraj

%\backmatter

\tableofcontents

\end{document}
