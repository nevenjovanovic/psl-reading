\documentclass[12pt]{article}
\usepackage{polyglossia}
\setdefaultlanguage{croatian}
\textwidth=7in
\textheight=9.5in
\topmargin=-1in
\headheight=0in
\headsep=.5in
\hoffset  -.85in

%\usepackage{fontspec}
%\usepackage{verse}
\defaultfontfeatures{Ligatures=TeX}
\setmainfont{Kyrtos}


\usepackage{hyperref}
\usepackage{enumitem}

\pagestyle{empty}

%\renewcommand{\thefootnote}{\fnsymbol{footnote}}
\begin{document}

\begin{center}
\textit{Prevođenje s latinskoga}, ISVU 51430, 6 ECTS\\
zimski semestar 2019/20, utorkom 8–9:30, srijedom 8–9:30
\end{center}

\setlength{\unitlength}{1in}

\begin{picture}(6,.1) 
\put(0,0) {\line(1,0){6.25}}         
\end{picture}

 

\renewcommand{\arraystretch}{2}

\vskip.15in
\noindent\textit{Nastavnik:} Neven Jovanović, F319, \texttt{neven.jovanovic@ffzg.hr}

%\vskip.25in
%\noindent\textit{Konzultacije:} (u ljetnom semestru) utorkom i srijedom od 10:30 do 11:30 i po dogovoru.

%\vskip.25in
%\noindent\textit{Omega:}  \url{https://omega.ffzg.hr/course/view.php?id=3032}\\
%\textit{Lozinka za upis}: \texttt{uak}

%\vskip.25in
%\noindent\textbf{Prerequisites:}\footnotemark
%A list of prerequisites.


%\footnotetext{Footnote text goes here.}

%\vspace*{.15in}

\section*{Plan rada – predavanja} 

\begin{center} \begin{minipage}{5.5in}
\begin{flushleft}

\hspace*{-0.5in} 0. (1. 10.) Predstavljanje kolegija i ciljeva

\hspace*{-0.5in} 1. (8. 10.) Primjer dobrog prijevoda

\hspace*{-0.5in} 2. (15. 10.) Primjer problematičnog prijevoda

\hspace*{-0.5in} 3. (22. 10.) Prema sistematiziranju problema prevođenja s latinskog - leksik

\hspace*{-0.5in} 4. (29. 10.) Prema sistematiziranju problema prevođenja s latinskog - sintaksa

\hspace*{-0.5in} 5. (5. 11.) Prema sistematiziranju problema prevođenja s latinskog - arhaiziranje i anakronizam

\hspace*{-0.5in} 6. (12. 11.) Teorija valjanog prijevoda – točnost i vjernost

\hspace*{-0.5in} 7. (19. 11.) Analiza primjera valjanog prijevoda

\hspace*{-0.5in} 8. (26. 11.) Samostalna analiza prijevoda – rekapitulacija dosadašnjih spoznaja

\hspace*{-0.5in}  9. (3. 12.) Samostalan valjani prijevod sintagmi

\hspace*{-0.5in}  10. (10. 12.) Samostalan valjani prijevod rečenica

\hspace*{-0.5in}  11. (17. 12.) Samostalan valjani prijevod teksta

\medskip

\hspace*{-0.5in}  \textit{~~~ Božićni i novogodišnji praznici}

\medskip

\hspace*{-0.5in}  12. (7. 1.) Zajednički rad – priprema prijevoda

\hspace*{-0.5in}  13. (14. 1.) Zajednički rad – kritika prijevoda

\hspace*{-0.5in}  14. (21. 1.) Zaključak


\end{flushleft}
\end{minipage}
\end{center}

\section*{Ciljevi} 

Definirati što je točan, a što vjeran prijevod, prema Peter Newmark, \textit{A textbook of translation,} 1987.

Navesti i prepoznati najčešće probleme pri prevođenju s latinskog (na području leksika, sintakse, anakronizama).

Vrednovati točnost i vjernost tuđeg prijevoda sintagmi, rečenica i čitavog kraćeg teksta.

U zajedničkom radu prirediti i prema uputama revidirati vlastiti prijevod kraćeg latinskog teksta.

Vlastiti prijevod predstaviti drugima i braniti svoja rješenja.

\section*{Zadaci} 

Samostalna lektira: Ciceron, Seneka, Terencije, Tibul, Horacije. Tjedna priprema pismenih prijevoda. Rasprava o prevodilačkim problemima u skupini i s kolegama sa Sveučilišta u Cincinnatiju.

\end{document}
