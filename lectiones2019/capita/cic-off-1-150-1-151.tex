
Iam de artificiis et quaestibus, qui liberales habendi, qui sordidi sint, haec fere accepimus. Primum improbantur ii quaestus, qui in odia hominum incurrunt, ut portitorum, ut faeneratorum. Illiberales autem et sordidi quaestus mercennariorum omnium, quorum operae, non quorum artes emuntur; est enim in illis ipsa merces auctoramentum servitutis. Sordidi etiam putandi, qui mercantur a mercatoribus, quod statim vendant; nihil enim proficiant, nisi admodum mentiantur; nec vero est quicquam turpius vanitate. Opificesque omnes in sordida arte versantur; nec enim quicquam ingenuum habere potest officina. Minimeque artes eae probandae, quae ministrae sunt voluptatum: ``Cetárii, lanií, coqui, fartóres, piscatóres,'' ut ait Terentius; adde huc, si placet, unguentarios, saltatores totumque ludum talarium.

Quibus autem artibus aut prudentia maior inest aut non mediocris utilitas quaeritur, ut medicina, ut architectura, ut doctrina rerum honestarum, eae sunt iis, quorum ordini conveniunt, honestae. Mercatura autem, si tenuis est, sordida putanda est; sin magna et copiosa, multa undique apportans multisque sine vanitate impertiens, non est admodum vituperanda, atque etiam, si satiata quaestu vel contenta potius, ut saepe ex alto in portum, ex ipso portu se in agros possessionesque contulit, videtur iure optimo posse laudari. Omnium autem rerum, ex quibus aliquid acquiritur, nihil est agri cultura melius, nihil uberius, nihil dulcius, nihil homine libero dignius; de qua quoniam in Catone Maiore satis multa diximus, illim assumes, quae ad hunc locum pertinebunt.



