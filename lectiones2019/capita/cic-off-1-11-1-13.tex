
Principio generi animantium omni est a natura tributum, ut se, vitam corpusque tueatur, declinet ea, quae nocitura videantur, omniaque, quae sint ad vivendum necessaria, anquirat et paret, ut pastum, ut latibula, ut alia generis eiusdem. Commune item animantium omnium est coniunctionis adpetitus procreandi causa et cura quaedam eorum, quae procreata sint; sed inter hominem et beluam hoc maxime interest, quod haec tantum, quantum sensu movetur, ad id solum, quod adest quodque praesens est, se accommodat paulum admodum sentiens praeteritum aut futurum; homo autem, quod rationis est particeps, per quam consequentia cernit, causas rerum videt earumque praegressus et quasi antecessiones non ignorat, similitudines comparat rebusque praesentibus adiungit atque annectit futuras, facile totius vitae cursum videt ad eamque degendam praeparat res necessarias.

Eademque natura vi rationis hominem conciliat homini et ad orationis et ad vitae societatem ingeneratque in primis praecipuum quendam amorem in eos, qui procreati sunt, impellitque, ut hominum coetus et celebrationes et esse et a se obiri velit ob easque causas studeat parare ea, quae suppeditent ad cultum et ad victum, nec sibi soli, sed coniugi, liberis ceterisque, quos caros habeat tuerique debeat; quae cura exsuscitat etiam animos et maiores ad rem gerendam facit.

In primisque hominis est propria veri inquisitio atque investigatio. Itaque cum sumus necessariis negotiis curisque vacui, tum avemus aliquid videre, audire, addiscere cognitionemque rerum aut occultarum aut admirabilium ad beate vivendum necessariam ducimus. Ex quo intellegitur, quod verum, simplex sincerumque sit, id esse naturae hominis aptissimum. Huic veri videndi cupiditati adiuncta est appetitio quaedam principatus, ut nemini parere animus bene informatus a natura velit nisi praecipienti aut docenti aut utilitatis causa iuste et legitime imperanti; ex quo magnitudo animi exsistit humanarumque rerum contemptio.

\subsection*{Cic. Off. 1, 18}


Ex quattuor autem locis, in quos honesti naturam vimque divisimus, primus ille, qui in veri cognitione consistit, maxime naturam attingit humanam. Omnes enim trahimur et ducimur ad cognitionis et scientiae cupiditatem, in qua excellere pulchrum putamus, labi autem, errare, nescire, decipi et malum et turpe ducimus. In hoc genere et naturali et honesto duo vitia vitanda sunt, unum, ne incognita pro cognitis habeamus iisque temere assentiamur; quod vitium effugere qui volet (omnes autem velle debent), adhibebit ad considerandas res et tempus et diligentiam.
