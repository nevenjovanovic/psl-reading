%\section*{O autoru}

Tibullus hac Elegia queritur appositum esse custodem amicae suae Deliae docetque qua ratione custodes fallat ut ipsi inter se possint amores suos conferre etiam praesente viro. 



%\poemtitle*{Elegia II.}

%Naslov prema izdanju




\settowidth{\versewidth}{Nec tamen huic credet coniunx tuus, ut mihi verax}
\begin{verse}[\versewidth]\poemlines{5}
\begin{altverse}
{\large
Adde merum vinoque novos conpesce dolores,\\
Occupet ut fessi lumina victa sopor,\\
Neu quisquam multo percussum tempora baccho\\
Excitet, infelix dum requiescit amor.\\
Nam posita est nostrae custodia saeva puellae,\\
Clauditur et dura ianua firma sera.\\
Ianua difficilis domini, te verberet imber,\\
Te Iovis imperio fulmina missa petant.\\
Ianua, iam pateas uni mihi, victa querelis,\\
Neu furtim verso cardine aperta sones.\\
Et mala siqua tibi dixit dementia nostra,\\
Ignoscas: capiti sint precor illa meo.\\
Te meminisse decet, quae plurima voce peregi\\
Supplice, cum posti florida serta darem.\\
Tu quoque ne timide custodes, Delia, falle,\\
Audendum est: fortes adiuvat ipsa Venus.\\
Illa favet, seu quis iuvenis nova limina temptat,\\
Seu reserat fixo dente puella fores;\\
Illa docet molli furtim derepere lecto,\\
Illa pedem nullo ponere posse sono,\\
Illa viro coram nutus conferre loquaces\\
Blandaque conpositis abdere verba notis.\\
Nec docet hoc omnes, sed quos nec inertia tardat\\
Nec vetat obscura surgere nocte timor.\\
En ego cum tenebris tota vagor anxius urbe,\\
$\langle$Securum tenebris me facit ipsa Venus$\rangle$\\
Nec sinit occurrat quisquam, qui corpora ferro\\
Volneret aut rapta praemia veste petat.\\
Quisquis amore tenetur, eat tutusque sacerque\\
Qualibet: insidias non timuisse decet.\\
Non mihi pigra nocent hibernae frigora noctis,\\
Non mihi, cum multa decidit imber aqua.\\
Non labor hic laedit, reseret modo Delia postes\\
Et vocet ad digiti me taciturna sonum.\\
Parcite luminibus, seu vir seu femina fiat\\
Obvia: celari volt sua furta Venus.\\
Neu strepitu terrete pedum neu quaerite nomen\\
Neu prope fulgenti lumina ferte face.\\
Siquis et inprudens adspexerit, occulat ille\\
Perque deos omnes se meminisse neget:\\
Nam fuerit quicumque loquax, is sanguine natam,\\
Is Venerem e rapido sentiet esse mari.\\
Nec tamen huic credet coniunx tuus, ut mihi verax\\
Pollicita est magico saga ministerio.\\
Hanc ego de caelo ducentem sidera vidi,\\
Fluminis haec rapidi carmine vertit iter,\\
Haec cantu finditque solum Manesque sepulcris\\
Elicit et tepido devocat ossa rogo;\\
Iam tenet infernas magico stridore catervas,\\*
Iam iubet adspersas lacte referre pedem.\\
%\newpage
Cum libet, haec tristi depellit nubila caelo,\\*
Cum libet, aestivo convocat orbe nives.\\
Sola tenere malas Medeae dicitur herbas,\\*
Sola feros Hecates perdomuisse canes.\\
Haec mihi conposuit cantus, quis fallere posses:\\
Ter cane, ter dictis despue carminibus.\\
Ille nihil poterit de nobis credere cuiquam,\\
Non sibi, si in molli viderit ipse toro.\\
Tu tamen abstineas aliis: nam cetera cernet\\
Omnia, de me uno sentiet ipse nihil.\\
Quid, credam? nempe haec eadem se dixit amores\\
Cantibus aut herbis solvere posse meos,\\
Et me lustravit taedis, et nocte serena\\
Concidit ad magicos hostia pulla deos.\\
Non ego, totus abesset amor, sed mutuus esset,\\
Orabam, nec te posse carere velim.\\
Ferreus ille fuit, qui, te cum posset habere,\\
Maluerit praedas stultus et arma sequi.\\
Ille licet Cilicum victas agat ante catervas,\\
Ponat et in capto Martia castra solo,\\
Totus et argento contextus, totus et auro\\
Insideat celeri conspiciendus equo,\\
Ipse boves mea si tecum modo Delia possim\\
Iungere et in solito pascere monte pecus,\\
Et te, dum liceat, teneris retinere lacertis,\\
Mollis et inculta sit mihi somnus humo.\\
Quid Tyrio recubare toro sine amore secundo\\
Prodest, cum fletu nox vigilanda venit?\\
Nam neque tum plumae nec stragula picta soporem\\
Nec sonitus placidae ducere posset aquae.\\
Num Veneris magnae violavi numina verbo,\\*
Et mea nunc poenas inpia lingua luit?\\
%\newpage
Num feror incestus sedes adiisse deorum\\*
Sertaque de sanctis deripuisse focis?\\
Non ego, si merui, dubitem procumbere templis\\*
Et dare sacratis oscula liminibus,\\
Non ego tellurem genibus perrepere supplex\\
Et miserum sancto tundere poste caput.\\
At tu, qui laetus rides mala nostra, caveto\\
Mox tibi: non uni saeviet usque deus.\\
Vidi ego, qui iuvenum miseros lusisset amores,\\
Post Veneris vinclis subdere colla senem\\
Et sibi blanditias tremula conponere voce\\
Et manibus canas fingere velle comas,\\
Stare nec ante fores puduit caraeve puellae\\
Ancillam medio detinuisse foro.\\
Hunc puer, hunc iuvenis turba circumterit arta,\\
Despuit in molles et sibi quisque sinus.\\
At mihi parce, Venus: semper tibi dedita servit\\
Mens mea: quid messes uris acerba tuas? \\

}
\end{altverse}
\end{verse}

\newpage

\section*{Textus cum paraphrasi}

{\large

\noindent Adde merum vinoque novos conpesce dolores, occupet ut fessi lumina victa sopor, neu quisquam multo percussum tempora baccho excitet, infelix dum requiescit amor.\\

}


\noindent Affer vinum, et mero seda recentes molestias, ut somnus defatigati oculos domitas teneat; neve ullus expergefaciat me imbutum caput largo Baccho, donec amor infortunatus cessat.\\

{\large

\noindent Nam posita est nostrae custodia saeva puellae, clauditur et dura ianua firma sera.\\

}


\noindent Adhibitae sunt enim crudeles excubiae puellae meae; et fores stabiles valido pessulo obserantur.\\

{\large

\noindent Ianua difficilis domini, te verberet imber, te Iovis imperio fulmina missa petant.\\

}


\noindent O molesta janua heri, te pluvia feriat, te fulmina Jovis jussu jacta percutiant. \\

{\large

\noindent Ianua, iam pateas uni mihi, victa querelis, neu furtim verso cardine aperta sones. Et mala siqua tibi dixit dementia nostra, ignoscas: capiti sint precor illa meo.\\

}


\noindent O janua, mihi soli aperiaris exorata querimoniis meis, et ne strideas converso cardine clam reserata; et condones, si qua maledicta locuta est tibi insania mea, opto, ut illa capiti meo eveniant. \\



{\large

\noindent Te meminisse decet, quae plurima voce peregi supplice, cum posti florida serta darem.\\

}


\noindent Aequum est te recordari innumera quae verbis precantibus profudi, dum floreas coronas tuis postibus adhiberem. \\

{\large

\noindent Tu quoque ne timide custodes, Delia, falle, audendum est: fortes adiuvat ipsa Venus.\\

}


\noindent Tu etiam, o Delia, ne cunctanter excipe excubitores tuos. Venus ipsa audentes sublevat. \\

{\large

\noindent Illa favet, seu quis iuvenis nova limina temptat, seu reserat fixo dente puella fores; illa docet molli furtim derepere lecto, illa pedem nullo ponere posse sono, illa viro coram nutus conferre loquaces blandaque conpositis abdere verba notis.\\

}


\noindent Наeс auxiliatur, sive juvenis aliquis novas aedes experitur, sive puella clavi inserta aperit ostium. Наeс ipsa erudit clam egredi e tenero cubili, et nullo strepitu posse passum figere, et communicare nutus loquaces marito praesente, et tegere mollia verba pactis signis. \\

{\large

\noindent Nec docet hoc omnes, sed quos nec inertia tardat nec vetat obscura surgere nocte timor. En ego cum tenebris tota vagor anxius urbe, $\langle$securum tenebris me facit ipsa Venus$\rangle$ nec sinit occurrat quisquam, qui corpora ferro volneret aut rapta praemia veste petat.\\

}


\noindent Neque cunctos ea de re erudit: ut quos nec segnities remoratur, nec metus prohibet expergisci nocte tenebrosa: пес patitur ut ullus obsistat, qui corpora gladio laedat, vel detracto indumento pretium poscat. \\



{\large

\noindent Quisquis amore tenetur, eat tutusque sacerque qualibet: insidias non timuisse decet.\\

}


\noindent Quicumque amore occupatur, ille securus abeat et inviolabilis quacumque voluerit; par est non metuisse pericula. \\

\newpage

{\large

\noindent Non mihi pigra nocent hibernae frigora noctis, non mihi, cum multa decidit imber aqua. Non labor hic laedit, reseret modo Delia postes et vocet ad digiti me taciturna sonum.\\

}


\noindent Mihi non officit tardum gelu noctis hyemalis, nec quando pluvia crebra aqua delabitur: haec molestia non angit me, dummodo Delia mihi januam aperiat, et ipsa tacens accersat me ad digitorum strepitum. \\



{\large

\noindent Parcite luminibus, seu vir seu femina fiat obvia: celari volt sua furta Venus.\\

}


\noindent Abstinete lucernis, sive homo, sive mulier, mihi occurras: Venus cupit abscondi furta sua. \\

{\large

\noindent Neu strepitu terrete pedum neu quaerite nomen neu prope fulgenti lumina ferte face.\\

}


\noindent Nec me pedum sonitu perterrefacite, nec rogate nomen, nec admovete mihi propius lucernas claro lumine. \\

{\large

\noindent Siquis et inprudens adspexerit, occulat ille perque deos omnes se meminisse neget: nam fuerit quicumque loquax, is sanguine natam, is Venerem e rapido sentiet esse mari.\\

}


\noindent Si aliquis incogitans me viderit, is rem celet, et per cunctos caelites inficietur se recordari. Quisquis enim garrulus extiterit, ille experietur Venerem e cruore procreatam esse, et e truculento pelago. \\

{\large

\noindent Nec tamen huic credet coniunx tuus, ut mihi verax pollicita est magico saga ministerio.\\

}


\noindent Neque tamen maritus tuus isti adhibebit fidem, ut promisit mihi maga veridica officio suo sortilego. \\

\newpage

{\large

\noindent Hanc ego de caelo ducentem sidera vidi, fluminis haec rapidi carmine vertit iter, haec cantu finditque solum Manesque sepulcris elicit et tepido devocat ossa rogo; iam tenet infernas magico stridore catervas, iam iubet adspersas lacte referre pedem.\\

}


\noindent Ego ipsam spectavi ab aethere astra devocantem; ipsa cantibus retorquet cursum velocis amnis. Ipsa carminibus terram disrumpit, umbrasque e tumulis excitat, et ossa deripit e bustis calidis. Modo concitat magico sonitu turbas Tartareas; easque modo lacte permulsas gradum reportare imperat. \\

{\large

\noindent Cum libet, haec tristi depellit nubila caelo, cum libet, aestivo convocat orbe nives.\\

}


\noindent Quando vult, ipsa ex aere obscuro nubes fugat, quando vult aestivo cursu nives congregat. \\

{\large

\noindent Sola tenere malas Medeae dicitur herbas, sola feros Hecates perdomuisse canes.\\

}


\noindent Ipsa una fertur cognoscere malefica Medeae gramina, una pervicisse truculentos Hecates canes. \\

{\large

\noindent Haec mihi conposuit cantus, quis fallere posses: ter cane, ter dictis despue carminibus.\\

}


\noindent Ipsa confecit mihi carmina, quibus maritum decipere posses. Ter canta; prolatis cantibus, ter inspue. \\

{\large

\noindent Ille nihil poterit de nobis credere cuiquam, non sibi, si in molli viderit ipse toro.\\

}


\noindent Ille nemini quicquam credere poterit de nobis, nec sibi ipsi, etsi nos idem in tenero lecto conspexerit.\\

\newpage

{\large

\noindent Tu tamen abstineas aliis: nam cetera cernet omnia, de me uno sentiet ipse nihil.\\

}


\noindent At tu ab aliis temperes; reliqua enim omnia videbit, de me solo nihil esse intelliget. \\

{\large

\noindent Quid, credam? nempe haec eadem se dixit amores cantibus aut herbis solvere posse meos, et me lustravit taedis, et nocte serena concidit ad magicos hostia pulla deos.\\

}


\noindent Quid ego existimem? Scilicet ea ipsa affirmavit se carminibus aut graminibus dissolvere posse amores meos. Me etiam facibus purgavit; et furca procubuit victima ad magica numina clara nocte.  \\

{\large

\noindent Non ego, totus abesset amor, sed mutuus esset, orabam, nec te posse carere velim.\\

}


\noindent Non ego precabar ut integer amor absisteret, at reciprocus esset, nec cupiam posse reipsa destitui. \\

{\large

\noindent Ferreus ille fuit, qui, te cum posset habere, maluerit praedas stultus et arma sequi.\\

}


\noindent Ille ferreus sit, qui, si te tenere queat, praeoptarit stolidus exuvias ac militiam sectari. \\

{\large

\noindent Ille licet Cilicum victas agat ante catervas, ponat et in capto Martia castra solo, totus et argento contextus, totus et auro insideat celeri conspiciendus equo, ipse boves mea si tecum modo Delia possim iungere et in solito pascere monte pecus, et te, dum liceat, teneris retinere lacertis, mollis et inculta sit mihi somnus humo.\\

}


\noindent Esto ille copias Cilicum fusas coram propellat, et bellica tentoria statuat in terra occupata: ac totus argento, et totus auro concinnatus, spectandus sedeat veloci equo. Dum tecum sim, о Delia mea, ipse tauros jugare queam, et gregem pascere in desertis collibus. Et mihi dulcis quies sit in humo aspera, modo fas mihi sit te suaviter amplecti. \\

{\large

\noindent Quid Tyrio recubare toro sine amore secundo prodest, cum fletu nox vigilanda venit?\\

}


\noindent Quid juvat quiescere in lecto Tyrio absque amore prospero, quando nox lacrymis vigilanda est? \\

{\large

\noindent Nam neque tum plumae nec stragula picta soporem nec sonitus placidae ducere posset aquae.\\

}


\noindent Tunc enim nec plumae, nec peristromata variegata, пес sonus lenis aquae somnum queat conciliare. \\

{\large

\noindent Num Veneris magnae violavi numina verbo, et mea nunc poenas inpia lingua luit?\\

}


\noindent An potestatem Veneris magnae dicto offendi? Et lingua improba jam supplicium pendit? \\

{\large

\noindent Num feror incestus sedes adiisse deorum sertaque de sanctis deripuisse focis?\\

}


\noindent An dicor Divorum templa impurus invisisse, et coronas e sacris aris detraxisse? \\

{\large

\noindent Non ego, si merui, dubitem procumbere templis et dare sacratis oscula liminibus, non ego tellurem genibus perrepere supplex et miserum sancto tundere poste caput.\\

}


\noindent Equidem non cuncter, si dignus sum, eorum fanis procidere, et augustis foribus oscula praebere. Non ego cuncter genibus terram supplex perreptare, et sacris postibus infelix caput allidere. \\

{\large

\noindent At tu, qui laetus rides mala nostra, caveto mox tibi: non uni saeviet usque deus.\\

}


\noindent Tu vero, qui gaudens ludis mea incommoda, prospicito modo tibi ipsi: idem Deus non perpetuo iratus erit. \\

{\large

\noindent Vidi ego, qui iuvenum miseros lusisset amores, post Veneris vinclis subdere colla senem et sibi blanditias tremula conponere voce et manibus canas fingere velle comas, stare nec ante fores puduit caraeve puellae ancillam medio detinuisse foro.\\

}


\noindent Ego ipse spectavi, qui infelices amores irrisisset, postea vetulum subjicere cervicem Veneris jugo, et concinnare sibi illecebras trementi voce, et manibus cupere medicare canos capillos. Nec erubuit haerere ad januam, aut media platea amicae dilectae famulam remorari. \\

{\large

\noindent Hunc puer, hunc iuvenis turba circumterit arta, despuit in molles et sibi quisque sinus.\\

}


\noindent Ipsum pueri, ipsum adolescentes denso coetu circummurmurant; et singuli sibi in tenerum gremium despuunt. \\

{\large

\noindent At mihi parce, Venus: semper tibi dedita servit mens mea: quid messes uris acerba tuas?\\

}

\noindent Tu vero, о Venus, ignosce mihi, animus meus tibi devotus perpetuo fuit addictus. Quare incendis aspera fruges tuas?\\

