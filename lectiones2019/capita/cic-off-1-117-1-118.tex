
Haec igitur omnia, cum quaerimus, quid deceat, complecti animo et cogitatione debemus; in primis autem constituendum est, quos nos et quales esse velimus et in quo genere vitae, quae deliberatio est omnium difficillima. Ineunte enim adulescentia, cum est maxima imbecillitas consilii, tum id sibi quisque genus aetatis degendae constituit, quod maxime adamavit; itaque ante implicatur aliquo certo genere cursuque vivendi, quam potuit, quod optimum esset, iudicare.

Nam quod Herculem Prodicus dicit, ut est apud Xenophontem, cum primum pubesceret, quod tempus a natura ad deligendum, quam quisque viam vivendi sit ingressurus, datum est, exisse in solitudinem atque ibi sedentem diu secum multumque dubitasse, cum duas cerneret vias, unam Voluptatis, alteram Virtutis, utram ingredi melius esset, hoc Herculi ``Iovis satu edito'' potuit fortasse contingere, nobis non item, qui imitamur, quos cuique visum est, atque ad eorum studia institutaque impellimur; plerumque autem parentium praeceptis imbuti ad eorum consuetudinem moremque deducimur; alii multitudinis iudicio feruntur, quaeque maiori parti pulcherrima videntur, ea maxime exoptant; non nulli tamen sive felicitate quadam sive bonitate naturae sive parentium disciplina rectam vitae secuti sunt viam.
