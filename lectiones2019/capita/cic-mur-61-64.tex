
et quoniam non est nobis haec oratio habenda aut in imperita multitudine aut in aliquo conventu agrestium, audacius paulo de studiis humanitatis quae et mihi et vobis nota et iucunda sunt disputabo. in M.~Catone, iudices, haec bona quae videmus divina et egregia ipsius scitote esse propria; quae non numquam requirimus, ea sunt omnia non a natura verum a magistro. fuit enim quidam summo ingenio vir, Zeno, cuius inventorum aemuli Stoici nominantur. huius sententiae sunt et praecepta eius modi. sapientem gratia numquam moveri, numquam cuiusquam delicto ignoscere; neminem misericordem esse nisi stultum et levem; viri non esse neque exorari neque placari; solos sapientes esse, si distortissimi sint, formosos, si mendicissimi, divites, si servitutem serviant, reges; nos autem qui sapientes non sumus fugitivos, exsules, hostis, insanos denique esse dicunt; omnia peccata esse paria; omne delictum scelus esse nefarium, nec minus delinquere eum qui gallum gallinaceum, cum opus non fuerit, quam eum qui patrem suffocaverit; sapientem nihil opinari, nullius rei paenitere, nulla in re falli, sententiam mutare numquam.


hoc homo ingeniosissimus, M.~Cato, auctoribus eruditissimis inductus adripuit, neque disputandi causa, ut magna pars, sed ita vivendi. petunt aliquid publicani; cave ne quicquam habeat momenti gratia. supplices aliqui veniunt miseri et calamitosi; sceleratus et nefarius fueris, si quicquam misericordia adductus feceris. fatetur aliquis se peccasse et sui delicti veniam petit; 'nefarium est facinus ignoscere.' at leve delictum est. 'omnia peccata sunt paria.' dixisti quippiam: 'fixum et statutum est.' non re ductus es sed opinione; 'sapiens nihil opinatur.' errasti aliqua in re; male dici putat. hac ex disciplina nobis illa sunt: 'dixi in senatu me nomen consularis candidati delaturum.' iratus dixisti. 'numquam' inquit 'sapiens irascitur.' at temporis causa. 'improbi' inquit 'hominis est mendacio fallere; mutare sententiam turpe est, exorari scelus, misereri flagitium.'

nostri autem illi — fatebor enim, Cato, me quoque in adulescentia diffisum ingenio meo quaesisse adiumenta doctrinae — nostri, inquam, illi a Platone et Aristotele, moderati homines et temperati, aiunt apud sapientem valere aliquando gratiam; viri boni esse misereri; distincta genera esse delictorum et disparis poenas; esse apud hominem constantem ignoscendi locum; ipsum sapientem saepe aliquid opinari quod nesciat, irasci non numquam, exorari eundem et placari, quod dixerit interdum, si ita rectius sit, mutare, de sententia decedere aliquando; omnis virtutes mediocritate quadam esse moderatas.

hos ad magistros si qua te fortuna, Cato, cum ista natura detulisset, non tu quidem vir melior esses nec fortior nec temperantior nec iustior — neque enim esse potes — sed paulo ad lenitatem propensior. non accusares nullis adductus inimicitiis, nulla lacessitus iniuria, pudentissimum hominem summa dignitate atque honestate praeditum; putares, cum in eiusdem anni custodia te atque L.~Murenam fortuna posuisset, aliquo te cum hoc rei publicae vinculo esse coniunctum; quod atrociter in senatu dixisti, aut non dixisses aut, si potuisses, mitiorem in partem interpretarere.
