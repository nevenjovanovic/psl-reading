
vellem di immortales fecissent, patres conscripti, ut vivo potius Ser.~Sulpicio gratias ageremus quam honores mortuo quaereremus. nec vero dubito quin, si ille vir legationem renuntiare potuisset, reditus eius et vobis gratus fuerit et rei publicae salutaris futurus, non quo L.~Philippo et L.~Pisoni aut studium aut cura defuerit in tanto officio tantoque munere, sed cum Ser.~Sulpicius aetate illis anteiret, sapientia omnibus, subito ereptus e causa totam legationem orbam et debilitatam reliquit. 

quod si cuiquam iustus honos habitus est in morte legato, in nullo iustior quam in Ser.~Sulpicio reperietur. ceteri qui in legatione mortem obierunt ad incertum vitae periculum sine ullo mortis metu profecti sunt: Ser.~Sulpicius cum aliqua perveniendi ad M.~Antonium spe profectus est, nulla revertendi. qui cum ita adfectus esset ut, si ad gravem valetudinem labor accessisset, sibi ipse diffideret, non recusavit quo minus vel extremo spiritu, si quam opem rei publicae ferre posset, experiretur. itaque non illum vis hiemis, non nives, non longitudo itineris, non asperitas viarum, non morbus ingravescens retardavit, cumque iam ad congressum conloquiumque eius pervenisset ad quem erat missus, in ipsa cura ac meditatione obeundi sui muneris excessit e vita.

\subsection{Cic. Phil. 9, 10-11}

reddite igitur, patres conscripti, ei vitam cui ademistis. vita enim mortuorum in memoria est posita vivorum. perficite ut is quem vos inscii ad mortem misistis immortalitatem habeat a vobis. cui si statuam in rostris decreto vestro statueritis, nulla eius legationem posteritatis obscurabit oblivio. nam reliqua Ser. Sulpici vita multis erit praeclarisque monumentis ad omnem memoriam commendata. semper illius gravitatem, constantiam, fidem, praestantem in re publica tuenda curam atque prudentiam omnium mortalium fama celebrabit. nec vero silebitur admirabilis quaedam et incredibilis ac paene divina eius in legibus interpretandis, aequitate explicanda scientia. omnes ex omni aetate qui in hac civitate intellegentiam iuris habuerunt si unum in locum conferantur, cum Ser.~Sulpicio non sint comparandi. nec enim ille magis iuris consultus quam iustitiae fuit. 

ita ea quae proficiscebantur a legibus et ab iure civili semper ad facilitatem aequitatemque referebat, neque instituere litium actiones malebat quam controversias tollere. ergo hoc statuae monumento non eget; habet alia maiora. haec enim statua mortis honestae testis erit, illa memoria vitae gloriosae, ut hoc magis monumentum grati senatus quam clari viri futurum sit. 
