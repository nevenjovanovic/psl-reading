%\section*{O autoru}

Poeta incertum se dicit, quis Deus sub humana Augusti specie Divi Julii necem vindicaverit. (Non potuit Horatius invocare Deum ultorem necis Caesarianae, miles ipse Bruti: sed blanditur Augusto post bella civilia reddenti tranquillitatem: et pertinet forte, quod pulchre conjecit Sanadonus, neque Bentleianis rationibus repugnat, ad illud ipsum tempus, quo Augusti nomen, quod habet vetns inscriptio hujus carminis, suscepit. Vid.\ Dio LIII.\ 20.\ p.\ 510.\ c.\ et quae dicentur ad vs.\ 29. Gesn.)

\poemtitle*{Ad Augustum Caesarem.}

%Naslov prema izdanju




\settowidth{\versewidth}{sunt quos curriculo pulverem Olympicum}
\begin{verse}[\versewidth]\poemlines{5}
\indentpattern{0001}
\begin{patverse*}
%\begin{altverse}
{\large
Iam satis terris nivis atque dirae\\
grandinis misit pater et rubente 	\\
dextera sacras iaculatus arcis 	\\
  terruit urbem, 	\\
terruit gentis, grave ne rediret\\
saeculum Pyrrhae nova monstra questae, 	\\
omne cum Proteus pecus egit altos 	\\
  visere montis 	\\
piscium et summa genus haesit ulmo, 	\\
nota quae sedes fuerat columbis,\\
et superiecto pavidae natarunt 	\\
  aequore dammae. 	\\
vidimus flavom Tiberim retortis 	\\
litore Etrusco violenter undis 	\\
ire deiectum monumenta regis\\
  templaque Vestae, 	\\
Iliae dum se nimium querenti 	\\*
iactat ultorem, vagus et sinistra 	\\
labitur ripa Iove non probante u- 	\\
  xorius amnis. 	\\
audiet civis acuisse ferrum, 	\\
quo graves Persae melius perirent, 	\\
audiet pugnas vitio parentum 	\\
  rara iuventus. 	\\
quem vocet divum populus ruentis 	\\
imperi rebus? prece qua fatigent 	\\
virgines sanctae minus audientem 	\\
  carmina Vestam? 	\\
cui dabit partis scelus expiandi 	\\
Iuppiter? tandem venias precamur 	\\
nube candentis umeros amictus 	\\
  augur Apollo; 	\\
sive tu mavis, Erycina ridens, 	\\
quam Iocus circum volat et Cupido; 	\\
sive neglectum genus et nepotes 	\\
  respicis auctor, 	\\
heu nimis longo satiate ludo, 	\\
quem iuvat clamor galeaeque leves 	\\
acer et Marsi peditis cruentum 	\\
  voltus in hostem; 	\\
sive mutata iuvenem figura 	\\
ales in terris imitaris almae 	\\
filius Maiae patiens vocari 	\\
  Caesaris ultor, 	\\
serus in caelum redeas diuque 	\\
laetus intersis populo Quirini, 	\\
neve te nostris vitiis iniquum 	\\
  ocior aura 	\\
tollat: hic magnos potius triumphos, 	\\*
hic ames dici pater atque princeps, 	\\
neu sinas Medos equitare inultos 	\\
  te duce, Caesar.\\
  
}
\end{patverse*}
\end{verse}

%\newpage

\section*{Paraphrasis}

{
\setlength{\parindent}{0pt}

Jupiter jam in terras immisit sat nivis et horrendae grandinis, vibransque fulmina in sacras arces manu flammata, Romae terrorem incussit. Alias etiam nationes fecit timere ne rediret triste tempus Pyrrhae conqnerentis ob prodigia inaudita: quando Proteus totum armentum duxit in excelsos montes, atque piseium genus adhaesit ulmi fastigio; qui locus fuerat columbis cognitus: nec non damae timidi nataverunt in mari superfuso. 

Aspeximus flavum Tiberim ire prostratum monumenta Regis Numae, et aedem Vestae, aquis magno impetu reflexis a ripa Tusciam spectante: dum hic fluvius, Jove indignante, nimium indulgens uxori Iliae dolenti praeter modum, se vindicem ostentat atque in laevum littus errat exundans. 

Juvenes pauci culpa parentum audient aliquando Romanos strinxisse gladios, quibus juste magis confoderentur Persae graves; discentque bella civilia. 

Quem Deorum populus invocet labente Republica? Quibus votis instabunt Virgines sacratae apud Vestam obsecrationes minime suscipientem? Cui Jupiter dabit munus eluendi crimen? 

o Phoebe fatidice, obsecramus ut succurras, albos humeros nube velatos habens: seu vis potius adesse, o blanda Venus, circa quam Joci et Amores volitant: seu abjectam prolem atque posteros respicis, o Mars Romanae gentis parens, eheu bello nimis diuturno satiate, qui gaudes vociferationibus, et galeis politis, atque aspectu Mauri peditis erga saevum adversarium feroci. Sive immutata specie adolescentem exhibes in terra, tu alatus filius benignae Maiae, sinens te dici vindicem Julii Caesaris; tarde remigres im caelum, diuque maneas cum Populo Romano, neque celerior ventus rapiat te nostris criminibus infensum. Ama potius hic ingentes triumphos, et appellationem parentis auctorisque; neque patiaris, o Caesar, te imperante, Medos impune in equis vagari.

}
