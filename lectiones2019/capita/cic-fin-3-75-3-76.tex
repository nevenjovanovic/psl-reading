
quam gravis vero, quam magnifica, quam constans conficitur persona sapientis! qui, cum ratio docuerit, quod honestum esset, id esse solum bonum, semper sit necesse est beatus vereque omnia ista nomina possideat, quae irrideri ab inperitis solent. rectius enim appellabitur rex quam Tarquinius, qui nec se nec suos regere potuit, rectius magister populi — is enim est dictator — quam Sulla, qui trium pestiferorum vitiorum, luxuriae, avaritiae, crudelitatis, magister fuit, rectius dives quam Crassus, qui nisi eguisset, numquam Euphraten nulla belli causa transire voluisset. recte eius omnia dicentur, qui scit uti solus omnibus, recte etiam pulcher appellabitur — animi enim liniamenta sunt pulchriora quam corporis —, recte solus liber nec dominationi cuiusquam parens nec oboediens cupiditati, recte invictus, cuius etiamsi corpus constringatur, animo tamen vincula inici nulla possint, nec expectet ullum tempus aetatis, uti tum denique iudicetur beatusne fuerit, cum extremum vitae diem morte confecerit, quod ille unus e septem sapientibus non sapienter Croesum monuit; nam si beatus umquam fuisset, beatam vitam usque ad illum a Cyro extructum rogum pertulisset. quod si ita est, ut neque quisquam nisi bonus vir et omnes boni beati sint, quid philosophia magis colendum aut quid est virtute divinius?
