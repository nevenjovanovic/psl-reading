
Duodequadraginta annos tyrannus Syracusanorum fuit Dionysius, cum quinque et viginti natus annos dominatum occupavisset. qua pulchritudine urbem, quibus autem opibus praeditam servitute oppressam tenuit civitatem! atqui de hoc homine a bonis auctoribus sic scriptum accepimus, summam fuisse eius in victu temperantiam in rebusque gerundis virum acrem et industrium, eundem tamen maleficum natura et iniustum; ex quo omnibus bene veritatem intuentibus videri necesse est miserrimum. ea enim ipsa, quae concupierat, ne tum quidem, cum omnia se posse censebat, consequebatur.

qui cum esset bonis parentibus atque honesto loco natus — etsi id quidem alius alio modo tradidit — abundaretque et aequalium familiaritatibus et consuetudine propinquorum, haberet etiam more Graeciae quosdam adulescentis amore coniunctos, credebat eorum nemini, sed is quos ex familiis locupletium servos delegerat, quibus nomen servitutis ipse detraxerat, et quibusdam convenis et feris barbaris corporis custodiam committebat. ita propter iniustam dominatus cupiditatem in carcerem quodam modo ipse se incluserat. quin etiam ne tonsori collum committeret, tondere filias suas docuit. ita sordido ancillarique artificio regiae virgines ut tonstriculae tondebant barbam et capillum patris. et tamen ab is ipsis, cum iam essent adultae, ferrum removit instituitque, ut candentibus iuglandium putaminibus barbam sibi et capillum adurerent.

cumque duas uxores haberet, Aristomachen civem suam, Doridem autem Locrensem, sic noctu ad eas ventitabat, ut omnia specularetur et perscrutaretur ante. et cum fossam latam cubiculari lecto circumdedisset eiusque fossae transitum ponticulo ligneo coniunxisset, eum ipsum, cum forem cubiculi clauserat, detorquebat. idemque cum in communibus suggestis consistere non auderet, contionari ex turri alta solebat.

atque is cum pila ludere vellet — studiose enim id factitabat — tunicamque poneret, adulescentulo, quem amabat, tradidisse gladium dicitur. hic cum quidam familiaris iocans dixisset: ``huic quidem certe vitam tuam committis'', adrisissetque adulescens, utrumque iussit interfici, alterum, quia viam demonstravisset interimendi sui, alterum, quia dictum id risu adprobavisset. atque eo facto sic doluit, nihil ut tulerit gravius in vita; quem enim vehementer amarat, occiderat. sic distrahuntur in contrarias partis impotentium cupiditates. cum huic obsecutus sis, illi est repugnandum.

Quamquam hic quidem tyrannus ipse iudicavit, quam esset beatus. nam cum quidam ex eius adsentatoribus, Damocles, commemoraret in sermone copias eius, opes, maiestatem dominatus, rerum abundantiam, magnificentiam aedium regiarum negaretque umquam beatiorem quemquam fuisse, ``visne igitur'' inquit, ``o Damocle, quoniam te haec vita delectat, ipse eam degustare et fortunam experiri meam?''

cum se ille cupere dixisset, conlocari iussit hominem in aureo lecto strato pulcherrimo textili stragulo, magnificis operibus picto, abacosque compluris ornavit argento auroque caelato. tum ad mensam eximia forma pueros delectos iussit consistere eosque nutum illius intuentis diligenter ministrare.

aderant unguenta coronae, incendebantur odores, mensae conquisitissimis epulis extruebantur. fortunatus sibi Damocles videbatur. in hoc medio apparatu fulgentem gladium e lacunari saeta equina aptum demitti iussit, ut impenderet illius beati cervicibus. itaque nec pulchros illos ministratores aspiciebat nec plenum artis argentum nec manum porrigebat in mensam; iam ipsae defluebant coronae; denique exoravit tyrannum, ut abire liceret, quod iam beatus nollet esse. satisne videtur declarasse Dionysius nihil esse ei beatum, cui semper aliqui terror impendeat? atque ei ne integrum quidem erat, ut ad iustitiam remigraret, civibus libertatem et iura redderet; is enim se adulescens inprovida aetate inretierat erratis eaque commiserat, ut salvus esse non posset, si sanus esse coepisset.

\subsection*{Cic. Tusc. 5, 64-66}


ex eadem urbe humilem homunculum a pulvere et radio excitabo, qui multis annis post fuit, Archimedem. cuius ego quaestor ignoratum ab Syracusanis, cum esse omnino negarent, saeptum undique et vestitum vepribus et dumetis indagavi sepulcrum. tenebam enim quosdam senariolos, quos in eius monumento esse inscriptos acceperam, qui declarabant in summo sepulcro sphaeram esse positam cum cylindro.

ego autem cum omnia conlustrarem oculis — est enim ad portas Agragantinas magna frequentia sepulcrorum —, animum adverti columellam non multum e dumis eminentem, in qua inerat sphaerae figura et cylindri. atque ego statim Syracusanis — erant autem principes mecum — dixi me illud ipsum arbitrari esse, quod quaererem. inmissi cum falcibus multi purgarunt et aperuerunt locum.

quo cum patefactus esset aditus, ad adversam basim accessimus. apparebat epigramma exesis posterioribus partibus versiculorum dimidiatum fere. ita nobilissima Graeciae civitas, quondam vero etiam doctissima, sui civis unius acutissimi monumentum ignorasset, nisi ab homine Arpinate didicisset. 

sed redeat, unde aberravit oratio: quis est omnium, qui modo cum Musis, id est cum humanitate et cum doctrina, habeat aliquod commercium, qui se non hunc mathematicum malit quam illum tyrannum? si vitae modum actionemque quaerimus, alterius mens rationibus agitandis exquirendisque alebatur cum oblectatione sollertiae, qui est unus suavissimus pastus animorum, alterius in caede et iniuriis cum et diurno et nocturno metu. age confer Democritum, Pythagoram, Anaxagoram: quae regna, quas opes studiis eorum et delectationibus antepones?
