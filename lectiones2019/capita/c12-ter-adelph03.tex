
\poemtitle*{Scaena I. Sannio Aeschinus (Bacchis) Parmeno.}


\settowidth{\versewidth}{aliquid. vah quemquamne hominem in animo instituere aut xxxx xxxx xxxx}
\begin{verse}[\versewidth]\poemlines{5}
\setverselinenums{155}{155}
%{\large

\textit{Sannio:} Obsecro, populares, ferte misero atque innocenti auxilium,\\
subvenite inopi. \textit{Aeschinus:} otiose: nunciam ilico hic consiste.\\
quid respectas? nil periclist: numquam dum ego adero hic te tanget.\\
\textit{Sannio:} ego ĭstam invitis omnibus.\\
    \textit{Aeschinus:} quamquamst scelestu' non committet hodie umquam iterum ut vapulet.\\
\textit{Sannio:} Aeschine, audi ne te ignarum fuisse dicas meorum morum:\\
leno ego sum. \textit{Aeschinus:} scio. \textit{Sannio:} at ita ut usquam fuit fide quisquam optuma.\\
tu quod te posteriu' purges hanc iniuriam mi nolle\\
factam esse, hui(u)s non faciam. crede hoc, ego meum ius persequar\\
    neque tu verbis solves umquam quod mihi re male feceris.\\
    novi ego vostra haec: ``nollem factum: iusiurandum dabitur te esse\\
    indignum iniuria hac'' – indignis quom egomet sim acceptus modis.\\
\textit{Aeschinus:} abĭ prae strenue ac forěs aperi. \textit{Sannio:} ceterum hoc nihili facis?\\
\textit{Aeschinus:} ĭ ĭntro nunciăm. \textit{Sannio:} enĭm non sinam. \textit{Aeschinus:} accede illuc, Parmeno\\
    (nimium istoc abisti), hic propter hunc adsiste: em sic volo.\\
    cavě nunciam oculos a meĭs oculis quoquam demoveas tuos\\
    ne mora sit, si innuerim, quin pugnu' continuo in mala haereat.\\
    \textit{Sannio:} istuc volo ergo ipsum experiri. \textit{Aeschinus:} em serva. \textit{Parmeno:} omitte mulierem.\\
    \textit{Sannio:} o facinus indignum! \textit{Aeschinus:} geminabit nisi caves. \textit{Sannio:} ei, miseriam!\\
    \textit{Aeschinus:} non innueram; verum in ĭstam partem potiu' peccato tamen.\\
    i nunciam. – \textit{Sannio:} quid hŏc reist? regnumne, Aeschine, hic tu possides?\\
    \textit{Aeschinus:} si possiderem, ornatus esses ex tuis virtutibus.\\
    \textit{Sannio:} quid tibi rei mecumst? \textit{Aeschinus:} nil. \textit{Sa.:} quid? nostin qui sim? \textit{Ae.:} non desidero.\\
    \textit{Sannio:} tetigin tuĭ quicquam? \textit{Aeschinus:} si attigisses, ferres infortunium.\\
    \textit{Sannio:} qui tibi magis licet meam habere pro qua ego argentum dedi?\\
    responde. \textit{Aeschinus:} ante aedis non fecisse erĭt melius hic convicium;\\
    nam si molestu' pergis esse, iam intro abripiere atque ibi\\
    usque ad necem operiere loris. \textit{Sannio:} loris liber? \textit{Aeschinus:} sic erit.\\
    \textit{Sannio:} ŏ hominem inpurum! hicin libertatem aiunt esse aequam omnibus?\\
    \textit{Aeschinus:} si sati' iam debacchatus es, leno, audi si vis nunciam.\\
     \textit{Sannio:} egŏn debacchatu' sum autem an tŭ ĭn me? Ae.: mitte ista atque ad rem redi.\\
    \textit{Sannio:} quam rem? quo redeam? \textit{Aeschinus:} iamne me vis dicere id quod ăd te attinet?\\
    \textit{Sannio:} cupio, modo aequi aliquid. \textit{Aeschinus:} vah leno iniqua me non volt loqui.\\
    \textit{Sannio:} leno sum, fateor, pernicies communis adulescentium,\\
    periuru', pesti'; taměn tibi a me nulla est orta iniuria.\\
     \textit{Aeschinus:} nam hercle etiam hoc restat. \textit{Sannio:} illuc quaeso redĭ quo coepisti, Aeschine.\\
    \textit{Aeschinus:} minis viginti tŭ ĭllam emisti (quae res tibi vortat male!):\\
    argenti tantum dabitur. \textit{Sannio:} quid si ego tibi ĭllam nolo vendere?\\
    coges me? \textit{Aeschinus:} minime. \textit{Sannio:} namque id metui. Ae.: neque vendundam censeo\\
    quae liberast; nam ego liberali illam adsero causa manu.\\
    nunc vide utrum vis, argentum accipere an causam meditari tuam.\\
    delibera hoc dum ego redeo, leno.\\

    \textit{Aeschinus abit.}

                \textit{Sannio:} pro supreme Iuppiter,\\
    minime miror qui insanire occipiunt ex iniuria.\\
    domŏ me eripuit, verberavit; mě ĭnvito abduxit meam\\
    (ob male facta haec tantidem emptam postulat sibi tradier);\\
    homini misero plus quingentos colaphos infregit mihi.\\
    verum enĭm quando bene promeruit, fiat: suom ius postulat.\\
    age, iam cupio si modo ărgentum reddat. sed ego hŏc hariolor:\\
    ubi me dixero dare tanti, testis faciet ilico\\
    vendidisse me; dě ărgento – somnium: ``mox; cras redi.''\\
    id quoque possum ferre si modo reddat, quamquam iniuriumst.\\
    verum cogito id quod res est: quando eum quaestum occeperis,\\
    accipiunda et mussitanda iniuria adulescentiumst.\\*
    sed nemo dabit – : frustra egomet mecum has rationes puto.\\*
%}
\end{verse}

%\newpage

\section*{Paraphrasis}

{
\setlength{\parindent}{0pt}

\textit{Sa.} Per Deos oro, cives, auxiliamini infelici atque innocenti, subsidium afferte homini omnium ope privato. 

\textit{Aes.} Parmeno, jam nunc hic constanter et otiose mane. Quid toties aspicis, mulier? (fidicina,) nullum hic tibi periculum est; nunquam Sannio leno, dum hic praesens ero, te attinget. 

\textit{Sa.} Ego istam contra voluntatem omnium obsistentium retinebo. 

\textit{Aes.} Quamvis flagitiosus est, et nequam, nunquam quicquam hodie faciet, cur iterum a me caedatur. 

\textit{Sa.} Ausculta, Aeschine, ne te excuses, quod ingenium et institutum vitae meae haud noveris; ego leno sum.

\textit{Aes.} Probe id scio. 

\textit{Sa.} At sum talis, ut nullus usquam fidei magis observator fuerit: hoc scire te velim; nihil ducam, si ut te excuses, dixeris te invitum hanc mihi injuriam intulisse, te poenitere facti. Hoc mihi satisfactum non erit; nihilominus actionem, quae mihi competit, intendam; neque tu mihi unquam verbis satisfacies, quod facto mihi damnum et injuriam feceris. Mihi nota sunt vestra subterfugia et remedia juris, nimirum ista; paenitet me fecisse, jurejurando dices me non meritum esse, cui haec injuria fieret: haec mihi non satis erunt, cum aliter tractatus fuerim ac meritus sum. 

\textit{Aes.} Abi ante, Parmeno, confidenter; et januam aperi. 

\textit{Sa.} Atqui nihil efficis. 

\textit{Aes.} Tu, puella, intra nunc jam domum. 

\textit{Sa.} At enim non permittam ut intret. 

\textit{Aes.} Accede huc, Parmeno, nimium hinc secessisti. Hic prope hunc mane. Hem. Sic volo facias; vide ne oculos tuos avertas a meis oculis, Parmeno, ne in mora sis, si nutu oculorum significavero, quin pugnis malae lenunis contundantur. 

\textit{Sa.} Istud igitur volo ipsum, quod minaris, experiri.

\textit{Aes.} Hem, Parmeno, custodi puellam; tu, leno, omitte illam rapere. 

\textit{Sa.} O factum iniquum! 

\textit{Aes.} Duplicabit plagas Parmeno, nisi prospicis tibi. 

\textit{Sa.} O me miserum! 

\textit{Aes.} Non innueram, Parmeno, ut Sannioni alterum colaphum infringeres; verum quod tu cecideris eum, satius factum esse duco, et potius mihi, gratumque est, quam si non fecisses. Abi nunc jam cum virgine, Parmeno. 

\textit{Sa.} Quidnam hoc facti sibi vult? Regemne hic te gerere audes Athenis, libera civitate? 

\textit{Aes.} Si Rex hic essem, haberes tuis virtutibus digna praemia.

\textit{Sa.} Quid negotii est, quamobrem me caedis? 

\textit{Aes.} Nihil. 

\textit{Sa.} Quid? num tibi notus sum? 

\textit{Aes.} Non valde cupio te mihi notum esse. 

\textit{Sa.} Quidquamne ex tuis rebus attrectavi? 

\textit{Aes.} Si attigisses tantum, male et infeliciter id tibi cederet.

\textit{Sa.} Si mihi non permissum est quae tua sunt, attingere, cur tibi liceat puellam quam emi, retinere? responde.

\textit{Aes.} Utilius tibi erit, hic ante aedes publice injuriam mihi vociferando non fecisse: nam si persistis molestiam mihi exhibere, intro domum sublatus innumeris lororum plagis caederis, ad necem usque. 

\textit{Sa.} Me hominem liberum loris caedes? 

\textit{Aes.} Sic fiet. 

\textit{Sa.} O hominem flagitiis obscoenum! Itane libertate aequali Athenis omnes fruuntur? 

\textit{Aes.} Si satis bacchantis more insanivisti, leno, ausculta, si videtur, nunc jam tandem. 

\textit{Sa.} Egone bacchatus sum in te, an tu in me ? 

\textit{Aes.} Desine ista, atque ad rem, de qua inter nos agitur, redi. 

\textit{Sa.} Quam rem? quo redeam? 

\textit{Aes.} Jamne tibi placet ut dicam quod ad te spectat? 

\textit{Sa.} Volo, dummodo aequitatis aliquid habeat. 

\textit{Aes.} Vah! leno religiosus scilicet, me cupit aequa proponere. 

\textit{Sa.} Leno sum, haud nego, exitium juventutis, perjurus, pestis publica, et calamitas, tamen nulla coepta est a me tibi fieri injuria. 

\textit{Aes.} Nempe id unum tibi faciendum superest. 

\textit{Sa.} Rogo, Aeschine, ut pergas loqui de re ipsa, quae ad nos attinet, ut jam incepisti. 

\textit{Aes.} Minis viginti tu illam fidicinam emisti, quae res male tibi succedat: tantumdem pecuniae tibi reddetur. 

\textit{Sa.} Quid si ego illam nolo tibi vendere, an me vi compelles ad id faciendum?

\textit{Aes.} Minime. 

\textit{Sa.} Quasi vero id metuam. 

\textit{Aes.} Neque eam venumdari posse aio, quae libera est: nam ego illam vindico, et assero ejus libertatem actione, quae est de libertate. Nunc delibera, utrum vis facere, vel recipere viginti minas, quas pro pretio ejus dedisti, aut te praeparare ad judicium de libertate ad quod te provoco: hoc apud te constitue, donec ego huc revertar, leno. 

\textit{Sa.} Proh supreme Deorum Jupiter! minime miror insanos fieri per injuriam quamplurimos, cum ego insaniam ex injuria, quam accepi: domo me extraxit; cecidit; per vim, contra voluntatem meam servam mihi abripuit; pro his malefactis poscit, ut tanti ei hanc fidicinam vendam, quanti eam emi. Mihi homini misero plusquam quingentos colaphos inflixit: enimvero quia tam bene meritus est de me, fiat quod vult; postulat jus suum sibi tribui scilicet. Age, consentio, si modo pecuniam rependat; sed ego hoc divino, ubi me dixero puellam ei tradere tanti, quanti eam emi, statim testes vocabit, dicet viginti minis me illi vendidisse; nec postea competet mihi actio de vi propter ereptam virginem: quantum ad argentum spectat, spes nulla, aut certe vana erit instar somnii; non numerabitur; semper eludet procrastinatione, dicens subinde, cras ad me huc revertere, pretium solvam conventum. Has dilationes et moras possum etiam ferre, si modo tandem quod promisit praestet, et viginti minae mihi reponantur: quamvis injuriam mihi faciat, cogens me invitum vendere: atqui mihi propono id, quod verum est, et mecum sic loquor; quando tu leno, turpe lucrum facere coepisti, suscipienda et toleranda est cum silentio, ne verbum quidem mutiendo, contumelia adolescentium. Sed nullum mihi reddetur ab Aeschino argentum: frustra sum, et mecum ratiocinationes male subduco.

}
