%\section*{O autoru}



\section*{De itineribus et de lectione}

Lucilium, qui Senecae scripserat se lectioni librorum operam navare nec locorum mutationibus delectari, laudat et, ut in his inceptis pergat, hortatur.

Prope idem argumentum Ep.\ XXVIII est.

Addit egregia consilia de recte et utiliter legendis libris, quae in memoriam revocant illud Quintiliani Instit.\ Orat.\ X, 1, multa magis quam multorum lectione firmandam mentem et ducendum esse colorem: sive quod alio loco acute praecepit, non multa, sed multum legere oportere. Egregium tandem illud consilium, ut Lucilius sibi quotidie e lectione sua praeceptum aliquod efficax excerpat, a se ipso custodiri ait Seneca, ideoque finit epistolam dicto Epicureo: honestam rem esse laetam paupertatem.

Saepius revertitur Noster ad haec consilia de lectione recte instituenda. Cf.\ de Tranquill.\ Anim.\ IX, 4; Epist.\ XLV, LXXXVIII, etc.

Mature autem vitia, quae in legendis libris committerentur, fuisse observata a viris intelligentibus docent vel Aristarchi et Aristophanis Bysant.\ praecepta, sive indices librorum ad legendum fructuosissimorum, sub nomine canonum notissimi. Quibus tamen illa vitia non prorsus sublata esse ex studio hominum isto, quo videri quam esse malint, satis superque pateret, nisi adhuc Polybius III, 57, p.\ 513 sq.\ Tom.\ I, ed.\ Schweighaeus.\ de eo quaereretur. Sed haec obiter et quasi perfusorie hic tantum attingere licet.

\newpage

\section*{Pročitajte naglas latinski tekst.}

%Naslov prema izdanju

Sen. ep. 2

\medskip

{\large
\noindent Seneca Lucilio suo salutem

\medskip


\noindent Ex iis quae mihi scribis, et ex iis quae audio, bonam spem de te concipio; non discurris nec locorum mutationibus inquietaris. Aegri animi ista iactatio est. Primum argumentum conpositae mentis existimo posse consistere et secum morari.


Illud autem vide, ne ista lectio auctorum multorum et omnis generis voluminum habeat aliquid vagum et instabile. Certis ingeniis inmorari et innutriri oportet, si velis aliquid trahere, quod in animo fideliter sedeat. Nusquam est, qui ubique est. Vitam in peregrinatione exigentibus hoc evenit, ut multa hospitia habeant, nullas amicitias. Idem accidat necesse est iis, qui nullius se ingenio familiariter applicant, sed omnia cursim et properantes transmittunt.

Non prodest cibus nec corpori accedit, qui statim sumptus emittitur; nihil aeque sanitatem impedit quam remediorum crebra mutatio; non venit vulnus ad cicatricem, in quo medicamenta temptantur; non convalescit planta, quae saepe transfertur. Nihil tam utile est, ut in transitu prosit. 

Distringit librorum multitudo. Itaque cum legere non possis, quantum habueris, satis est habere, quantum legas.

``Sed modo,'' inquis, ``hunc librum evolvere volo, modo illum.'' Fastidientis stomachi est multa degustare; quae ubi varia sunt et diversa, inquinant, non alunt. Probatos itaque semper lege, et si quando ad alios deverti libuerit, ad priores redi. Aliquid cotidie adversus paupertatem, aliquid adversus mortem auxilii compara, nec minus adversus ceteras pestes; et cum multa percurreris, unum excerpe, quod illo die concoquas.

Hoc ipse quoque facio; ex pluribus, quae legi, aliquid adprehendo. Hodiernum hoc est, quod apud Epicurum nanctus sum; soleo enim et in aliena castra transire, non tamquam transfuga, sed tamquam explorator.

``Honesta,'' inquit, ``res est laeta paupertas.'' Illa vero non est paupertas, si laeta est. Non qui parum habet, sed qui plus cupit, pauper est. Quid enim refert, quantum illi in arca, quantum in horreis iaceat, quantum pascat aut feneret, si alieno inminet, si non adquisita sed adquirenda computat? 

Quis sit divitiarum modus, quaeris? Primus habere quod necesse est, proximus quod sat est. 

VALE.

}


%\section*{Analiza}

%1

%{\large
%\noindent Ita fac, mi Lucili; \\
