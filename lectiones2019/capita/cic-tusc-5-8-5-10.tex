

nec vero Atlans sustinere caelum nec Prometheus adfixus Caucaso nec stellatus Cepheus cum uxore genero filia traderetur, nisi caelestium divina cognitio nomen eorum ad errorem fabulae traduxisset. a quibus ducti deinceps omnes, qui in rerum contemplatione studia ponebant, sapientes et habebantur et nominabantur, idque eorum nomen usque ad Pythagorae manavit aetatem. quem, ut scribit auditor Platonis Ponticus Heraclides, vir doctus in primis, Phliuntem ferunt venisse, eumque cum Leonte, principe Phliasiorum, docte et copiose disseruisse quaedam. cuius ingenium et eloquentiam cum admiratus esset Leon, quaesivisse ex eo, qua maxime arte confideret; at illum: artem quidem se scire nullam, sed esse philosophum. admiratum Leontem novitatem nominis quaesivisse, quinam essent philosophi, et quid inter eos et reliquos interesset; Pythagoram autem respondisse similem sibi videri vitam hominum et mercatum eum, qui haberetur maxumo ludorum apparatu totius Graeciae celebritate; nam ut illic alii corporibus exercitatis gloriam et nobilitatem coronae peterent, alii emendi aut vendendi quaestu et lucro ducerentur, esset autem quoddam genus eorum, idque vel maxime ingenuum, qui nec plausum nec lucrum quaererent, sed visendi causa venirent studioseque perspicerent, quid ageretur et quo modo, item nos quasi in mercatus quandam celebritatem ex urbe aliqua sic in hanc vitam ex alia vita et natura profectos alios gloriae servire, alios pecuniae, raros esse quosdam, qui ceteris omnibus pro nihilo habitis rerum naturam studiose intuerentur; hos se appellare sapientiae studiosos — id est enim philosophos —; et ut illic liberalissimum esset spectare nihil sibi adquirentem, sic in vita longe omnibus studiis contemplationem rerum cognitionemque praestare.

Nec vero Pythagoras nominis solum inventor, sed rerum etiam ipsarum amplificator fuit. qui cum post hunc Phliasium sermonem in Italiam venisset, exornavit eam Graeciam, quae magna dicta est, et privatim et publice praestantissumis et institutis et artibus. cuius de disciplina aliud tempus fuerit fortasse dicendi. sed ab antiqua philosophia usque ad Socratem, qui Archelaum, Anaxagorae discipulum, audierat, numeri motusque tractabantur, et unde omnia orerentur quove reciderent, studioseque ab is siderum magnitudines intervalla cursus anquirebantur et cuncta caelestia. Socrates autem primus philosophiam devocavit e caelo et in urbibus conlocavit et in domus etiam introduxit et coëgit de vita et moribus rebusque bonis et malis quaerere.

