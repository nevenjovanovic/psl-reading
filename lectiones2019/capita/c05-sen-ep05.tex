%\section*{O autoru}



\section*{De philosophiae ostentatione et de vera philosophia}

Gratulatur amico, quod Philosophiae operam navat indefessam, sed monet simul uta ab ostentatione liberum se parestet: Philosophiae enim studium ad emendationem animi ac vitae adhibendum esse: qua una, si dissimilitudo a vulgo esse debeat, admiratio sit: reliqua, quae ad externam dissimilitudinem pertineant, tanquam ridicula et odiosa sint rejicienda. Emblema egregium ex Hecatone Stoico affert: Desines timere, si sperare desieris.

%\newpage

\section*{Pročitajte naglas latinski tekst.}

%Naslov prema izdanju

Sen. ep. 5

\medskip

{\large
\noindent Seneca Lucilio suo salutem

\medskip


\noindent Quod pertinaciter studes et omnibus omissis hoc unum agis, ut te meliorem cotidie facias, et probo et gaudeo, nec tantum hortor, ut perseveres, sed etiam rogo. Illud autem te admoneo, ne eorum more, qui non proficere sed conspici cupiunt, facias aliqua, quae in habitu tuo aut genere vitae notabilia sint. 

Asperum cultum et intonsum caput et neglegentiorem barbam et indictum argento odium et cubile humi positum, et quicquid aliud ambitio nempe perversa via sequitur, evita. Satis ipsum nomen philosophiae, etiam si modeste tractetur, invidiosum est; quid si nos hominum consuetudini coeperimus excerpere? Intus omnia dissimilia sint, frons populo nostra conveniat.

Non splendeat toga, ne sordeat quidem. Non habeamus argentum, in quod solidi auri caelatura descenderit, sed non putemus frugalitatis indicium auro argentoque caruisse. Id agamus, ut meliorem vitam sequamur quam vulgus, non ut contrariam; alioquin quos emendari volumus, fugamus a nobis et avertimus. Illud quoque efficimus, ut nihil imitari velint nostri, dum timent, ne imitanda sint omnia.

Hoc primum philosophia promittit, sensum communem, humanitatem et congregationem. A qua professione dissimilitudo nos separabit. Videamus, ne ista, per quae admirationem parare volumus, ridicula et odiosa sint. Nempe propositum nostrum est secundum naturam vivere; hoc contra naturam est, torquere corpus suum et faciles odisse munditias et squalorem adpetere et cibis non tantum vilibus uti sed taetris et horridis.

Quemadmodum desiderare delicatas res luxuriae est, ita usitatas et non magno parabiles fugere dementiae. Frugalitatem exigit philosophia, non poenam, potest autem esse non incompta frugalitas. Hic mihi modus placet: temperetur vita inter bonos mores et publicos; suspiciant omnes vitam nostram, sed agnoscant.

``Quid ergo? Eadem faciemus, quae ceteri? Nihil inter nos et illos intererit?'' Plurimum. Dissimiles esse nos vulgo sciat, qui inspexerit propius. Qui domum intraverit, nos potius miretur quam supellectilem nostram. Magnus ille est, qui fictilibus sic utitur quemadmodum argento. Nec ille minor est, qui sic argento utitur quemadmodum fictilibus. Infirmi animi est pati non posse divitias. 

Sed ut huius quoque diei lucellum tecum communicem, apud Hecatonem nostrum inveni cupiditatium finem etiam ad timoris remedia proficere. ``Desines,'' inquit, ``timere, si sperare desieris.'' Dices: ``Quomodo ista tam diversa pariter eunt?'' Ita est, mi Lucili: cum videantur dissidere, coniuncta sunt. Quemadmodum eadem catena et custodiam et militem copulat, sic ista, quae tam dissimilia sunt, pariter incedunt; spem metus sequitur Nec miror ista sic ire; utrumque pendentis animi est, utrumque futuri exspectatione solliciti.

Maxima autem utriusque causa est, quod non ad praesentia aptamur, sed cogitationes in longinqua praemittimus. Itaque providentia, maximum bonum condicionis humanae, in malum versa est.

Ferae pericula, quae vident, fugiunt; cum effugere, securae sunt; nos et venturo torquemur et praeterito. Multa bona nostra nobis nocent, timoris enim tormentum memoria reducit, providentia anticipat. Nemo tantum praesentibus miser est. VALE.

}


%\section*{Analiza}

%1

