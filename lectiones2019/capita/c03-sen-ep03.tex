%\section*{O autoru}



\section*{De eligendis amicis}

Familiariter reprehendit Lucilium quod amici nomine, ut vulgus soleat, usus sit, monetque eum demum vere ac proprie amicum esse quocum, si probatus fuerit, omnia nostra aeque ac nobiscum communicare possimus. Unde hoc nos proficere addit ut nec ipsi mali quid committamus et amicum constantem reddamus et fidelem. In quo et a vulgo sejungamus nos oportet, quippe qui nullo amici discrimine facto vel omnes amicorum loco habent vel amicitiae praestantiam ignorant. (Cf.\ Epist.\ IX, ubi alio eoque stoico more hunc locum persequitur.) 

Quae quidem observatio de hominum inconstantia similem eorum errorem in ejus mentem revocat, utpote qui industriam in circumcursione perpetua et quietem in omni vacuitate a motu ponunt. 

Satis laxo vinculo adhibetur Pomponii dictum clausulae loco, ejusdem fere argumenti. 

Caeterum et de amicitia omnes egisse auctores, qui de officiis exposuerint, constat. In Stoicis autem hac re tractanda clari fuere Posidonius, Hecato, Panaetius (Vide Diog.\ Laert.\ VII, 124): quibus et Cicero in libro egregio de Amicitia, dialogo praestantissimo Platonis, cui titulus est: Lysis, in partes vocato, idemtidem usus esse videtur.

\newpage

\section*{Pročitajte naglas latinski tekst.}

%Naslov prema izdanju

Sen. ep. 3

\medskip

{\large
\noindent Seneca Lucilio suo salutem

\medskip


\noindent Epistulas ad me perferendas tradidisti, ut scribis, amico tuo; deinde admones me, ne omnia cum eo ad te pertinentia communicem, quia non soleas ne ipse quidem id facere; ita in eadem epistula illum et dixisti amicum et negasti. Itaque si proprio illo verbo quasi publico usus es et sic illum amicum vocasti, quomodo omnes candidatos bonos viros dicimus, quomodo obvios, si nomen non succurrit, dominos salutamus, hac abierit.

Sed si aliquem amicum existimas, cui non tantundem credis quantum tibi, vehementer erras et non satis nosti vim verae amicitiae. Tu vero omnia cum amico delibera, sed de ipso prius. Post amicitiam credendum est, ante amicitiam iudicandum. Isti vero praepostero officia permiscent, qui contra praecepta Theophrasti, cum amaverunt, iudicant, et non amant, cum iudicaverunt. Diu cogita, an tibi in amicitiam aliquis recipiendus sit. Cum placuerit fieri, toto illum pectore admitte; tam audaciter cum illo loquere quam tecum.

Tu quidem ita vive, ut nihil tibi committas, nisi quod committere etiam inimico tuo possis; sed quia interveniunt quaedam, quae consuetudo fecit arcana, cum amico omnes curas, omnes cogitationes tuas misce. Fidelem si putaveris, facies. Nam quidam fallere docuerunt, dum timent falli, et illi ius peccandi suspicando fecerunt. Quid est, quare ego ulla verba coram amico meo retraham? Quid est, quare me coram illo non putem solum?

Quidam quae tantum amicis committenda sunt, obviis narrant et in quaslibet aures, quicquid illos urserit, exonerant. Quidam rursus etiam carissimorum conscientiam reformidant, et si possent, ne sibi quidem credituri interius premunt omne secretum. Neutrum faciendum est. Utrumque enim vitium est, et omnibus credere et nulli. Sed alterum honestius dixerim vitium, alterum tutius; sic utrosque reprehendas, et eos qui semper inquieti sunt, et eos qui semper quiescunt.

Nam illa tumultu gaudens non est industria, sed exagitatae mentis concursatio. Et haec non est quies, quae motum omnem molestiam iudicat, sed dissolutio et languor.

Itaque hoc, quod apud Pomponium legi, animo mandabitur: ``quidam adeo in latebras refugerunt, ut putent in turbido esse, quicquid in luce est.'' Inter se ista miscenda sunt, et quiescenti agendum et agenti quiescendum est. Cum rerum natura delibera; illa dicet tibi et diem fecisse se et noctem. VALE.

}


%\section*{Analiza}

%1

%{\large
%\noindent Ita fac, mi Lucili; \\
