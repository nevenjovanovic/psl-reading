
Illi vero non modo cum hostibus, verum etiam cum fluctibus, id quod fecit, dimicare melius fuit quam deserere consentientem Graeciam ad bellum barbaris inferendum.

Sed omittamus et fabulas et externa; ad rem factam nostramque veniamus. M.~Atilius Regulus cum consul iterum in Africa ex insidiis captus esset duce Xanthippo Lacedaemonio, imperatore autem patre Hannibalis Hamilcare, iuratus missus est ad senatum, ut, nisi redditi essent Poenis captivi nobiles quidam, rediret ipse Carthaginem. Is cum Romam venisset, utilitatis speciem videbat, sed eam, ut res declarat, falsam iudicavit; quae erat talis: manere in patria, esse domui suae cum uxore, cum liberis, quam calamitatem accepisset in bello, communem fortunae bellicae iudicantem tenere consularis dignitatis gradum. Quis haec negat esse utilia? quem censes? Magnitudo animi et fortitudo negat.

Num locupletiores quaeris auctores? Harum enim est virtutum proprium nihil extimescere, omnia humana despicere, nihil, quod homini accidere possit, intolerandum putare. Itaque quid fecit? In senatum venit, mandata exposuit, sententiam ne diceret recusavit, quam diu iure iurando hostium teneretur, non esse se senatorem. Atque illud etiam (``O stultum hominem,'' dixerit quispiam, ``et repugnantem utilitati suae!''), reddi captivos negavit esse utile; illos enim adulescentes esse et bonos duces, se iam confectum senectute. Cuius cum valuisset auctoritas, captivi retenti sunt, ipse Carthaginem rediit, neque eum caritas patriae retinuit nec suorum. Neque vero tum ignorabat se ad crudelissimum hostem et ad exquisita supplicia proficisci, sed ius iurandum conservandum putabat. Itaque tum, cum vigilando necabatur, erat in meliore causa, quam si domi senex captivus, periurus consularis remansisset.
