Quid? in motu et in statu corporis nihil inest, quod animadvertendum esse ipsa natura iudicet? quem ad modum quis ambulet, sedeat, qui ductus oris, qui vultus in quoque sit? nihilne est in his rebus, quod dignum libero aut indignum esse ducamus? nonne odio multos dignos putamus, qui quodam motu aut statu videntur naturae legem et modum contempsisse? et quoniam haec deducuntur de corpore, quid est cur non recte pulchritudo etiam ipsa propter se expetenda ducatur? nam si pravitatem inminutionemque corporis propter se fugiendam putamus, cur non etiam, ac fortasse magis, propter se formae dignitatem sequamur? et si turpitudinem fugimus in statu et motu corporis, quid est cur pulchritudinem non sequamur? atque etiam valitudinem, vires, vacuitatem doloris non propter utilitatem solum, sed etiam ipsas propter se expetemus. quoniam enim natura suis omnibus expleri partibus vult, hunc statum corporis per se ipsum expetit, qui est maxime e natura, quae tota perturbatur, si aut aegrum corpus est aut dolet aut caret viribus.

Videamus animi partes, quarum est conspectus illustrior; quae quo sunt excelsiores, eo dant clariora indicia naturae. tantus est igitur innatus in nobis cognitionis amor et scientiae, ut nemo dubitare possit quin ad eas res hominum natura nullo emolumento invitata rapiatur. videmusne ut pueri ne verberibus quidem a contemplandis rebus perquirendisque deterreantur? ut pulsi recurrant? ut aliquid scire se gaudeant? ut id aliis narrare gestiant? ut pompa, ludis atque eius modi spectaculis teneantur ob eamque rem vel famem et sitim perferant? quid vero? qui ingenuis studiis atque artibus delectantur, nonne videmus eos nec valitudinis nec rei familiaris habere rationem omniaque perpeti ipsa cognitione et scientia captos et cum maximis curis et laboribus compensare eam, quam ex discendo capiant, voluptatem?  ut mihi quidem Homerus huius modi quiddam vidisse videatur in iis, quae de Sirenum cantibus finxerit. neque enim vocum suavitate videntur aut novitate quadam et varietate cantandi revocare eos solitae, qui praetervehebantur, sed quia multa se scire profitebantur, ut homines ad earum saxa discendi cupiditate adhaerescerent. ita enim invitant Ulixem — nam verti, ut quaedam Homeri, sic istum ipsum locum —:
\begin{verse}
O decus Argolicum, quin puppim flectis, Ulixes,\\
Auribus ut nostros possis agnoscere cantus!\\
Nam nemo haec umquam est transvectus caerula cursu,\\ 
Quin prius adstiterit vocum dulcedine captus,\\
Post variis avido satiatus pectore musis\\
Doctior ad patrias lapsus pervenerit oras.\\
Nos grave certamen belli clademque tenemus,\\
Graecia quam Troiae divino numine vexit,\\
Omniaque e latis rerum vestigia terris.\\

\end{verse}

Vidit Homerus probari fabulam non posse, si cantiunculis tantus irretitus vir teneretur; scientiam pollicentur, quam non erat mirum sapientiae cupido patria esse cariorem. 

Atque omnia quidem scire, cuiuscumque modi sint, cupere curiosorum, duci vero maiorum rerum contemplatione ad cupiditatem scientiae summorum virorum est putandum. quem enim ardorem studii censetis fuisse in Archimede, qui dum in pulvere quaedam describit attentius, ne patriam quidem captam esse senserit? quantum Aristoxeni ingenium consumptum videmus in musicis? quo studio Aristophanem putamus aetatem in litteris duxisse? quid de Pythagora? quid de Platone aut de Democrito loquar? a quibus propter discendi cupiditatem videmus ultimas terras esse peragratas. quae qui non vident, nihil umquam magnum ac cognitione dignum amaverunt. Atque hoc loco, qui propter animi voluptates coli dicunt ea studia, quae dixi, non intellegunt idcirco esse ea propter se expetenda, quod nulla utilitate obiecta delectentur animi atque ipsa scientia, etiamsi incommodatura sit, gaudeant.
