% https://scaife.perseus.org/reader/urn:cts:latinLit:phi0474.phi008.perseus-lat2:1-41


\marginnote{1} si, quantum in agro locisque desertis audacia potest, tantum in foro atque in iudiciis impudentia valeret, non minus nunc in causa cederet A.~Caecina Sex.~Aebuti impudentiae, quam tum in vi facienda cessit audaciae. verum et illud considerati hominis esse putavit, qua de re iure disceptari oporteret, armis non contendere, et hoc constantis, quicum vi et armis certare noluisset, eum iure iudicioque superare. 
\marginnote{2} ac mihi quidem cum audax praecipue fuisse videtur Aebutius in convocandis hominibus et armandis, tum impudens in iudicio, non solum quod in iudicium venire ausus est — nam id quidem tametsi improbe fit in aperta re, tamen malitia est iam usitatum — sed quod non dubitavit id ipsum quod arguitur confiteri; nisi forte hoc rationis habuit, quoniam, si facta vis esset moribus, superior in possessione retinenda non fuisset, quia contra ius moremque facta sit, A.~Caecinam cum amicis metu perterritum profugisse; nunc quoque in iudicio si causa more institutoque omnium defendatur, nos inferiores in agendo non futuros; sin a consuetudine recedatur, se, quo impudentius egerit, hoc superiorem discessurum. quasi vero aut idem possit in iudicio improbitas quod in vi confidentia, aut nos non eo libentius tum audaciae cesserimus quo nunc impudentiae facilius obsisteremus.

\marginnote{3} itaque longe alia ratione, recuperatores, ad agendam causam hac actione venio atque initio veneram. tum enim nostrae causae spes erat posita in defensione mea, nunc in confessione adversarii, tum in nostris, nunc vero in illorum testibus; de quibus ego antea laborabam ne, si improbi essent, falsi aliquid dicerent, si probi existimarentur, quod dixissent probarent; nunc sum animo aequissimo. si enim sunt viri boni, me adiuvant, cum id iurati dicunt quod ego iniuratus insimulo; sin autem minus idonei, me non laedunt, cum eis sive creditur, creditur hoc ipsum quod nos arguimus, sive fides non habetur, de adversarii testium fide derogatur.

\marginnote{4} verum tamen cum illorum actionem causae considero, non video quid impudentius dici possit, cum autem vestram in iudicando dubitationem, vereor ne id quod videntur impudenter fecisse astute et callide fecerint. nam, si negassent vim hominibus armatis esse factam, facile honestissimis testibus in re perspicua tenerentur; sin confessi essent et id quod nullo tempore iure fieri potest tum ab se iure factum esse defenderent, sperarunt, id quod adsecuti sunt, se iniecturos vobis causam deliberandi et iudicandi iustam moram ac religionem. simul illud quod indignissimum est futurum arbitrati sunt, ut in hac causa non de improbitate Sex.~Aebuti, sed de iure civili iudicium fieri videretur.

\marginnote{5} qua in re, si mihi esset unius A.~Caecinae causa agenda, profiterer satis idoneum esse me defensorem, propterea quod fidem meam diligentiamque praestarem; quae cum sunt in actore causae, nihil est in re praesertim aperta ac simplici quod excellens ingenium requiratur. sed cum de eo mihi iure dicendum sit, quod pertineat ad omnis, quodque constitutum sit a maioribus, conservatum usque ad hoc tempus, quo sublato non solum pars aliqua iuris deminuta, sed etiam vis ea quae iuri maxime est adversaria iudicio confirmata esse videatur, video summi ingeni causam esse, non ut id demonstretur quod ante oculos est sed ne, si quis vobis error in tanta re sit obiectus, omnes potius me arbitrentur causae quam vos religioni vestrae defuisse.

\marginnote{6} quamquam ego mihi sic persuadeo, recuperatores, non vos tam propter iuris obscuram dubiamque rationem bis iam de eadem causa dubitasse quam, quod videtur ad summam illius existimationem hoc iudicium pertinere, moram ad condemnandum quaesisse simul et illi spatium ad sese conligendum dedisse. quod quoniam iam in consuetudinem venit et id viri boni vestri similes in iudicando faciunt, reprehendendum fortasse minus, querendum vero magis etiam videtur, ideo quod omnia iudicia aut distrahendarum controversiarum aut puniendorum maleficiorum causa reperta sunt, quorum alterum levius est, propterea quod et minus laedit et persaepe disceptatore domestico diiudicatur, alterum est vehementissimum, quod et ad graviores res pertinet et non honorariam operam amici, sed severitatem iudicis ac vim requirit. \marginnote{7} quod est gravius, et cuius rei causa maxime iudicia constituta sunt, id iam mala consuetudine dissolutum est. nam ut quaeque res est turpissima, sic maxime et maturissime vindicanda est, at eadem, quia existimationis periculum est, tardissime iudicatur. qui igitur convenit, quae causa fuerit ad constituendum iudicium, eandem moram esse ad iudicandum? si quis quod spopondit, qua in re verbo se uno obligavit, id non facit, maturo iudicio sine ulla religione iudicis condemnatur; qui per tutelam aut societatem aut rem mandatam aut fiduciae rationem fraudavit quempiam, in eo quo delictum maius est, eo poena est tardior? \marginnote{8} »est enim turpe iudicium.« ex facto quidem turpi. videte igitur quam inique accidat, quia res indigna sit, ideo turpem existimationem sequi; quia turpis existimatio sequatur, ideo rem indignam non vindicari.

ac si qui mihi hoc iudex recuperatorve dicat: »potuisti enim leviore actione confligere, potuisti ad tuum ius faciliore et commodiore iudicio pervenire; qua re aut muta actionem aut noli mihi instare ut iudicem tamen,« is aut timidior videatur quam fortem, aut cupidior quam sapientem iudicem esse aequum est, si aut mihi praescribat quem ad modum meum ius persequar, aut ipse id quod ad se delatum sit non audeat iudicare. etenim si praetor is qui iudicia dat numquam petitori praestituit qua actione illum uti velit, videte quam iniquum sit constituta iam re iudicem quid agi potuerit aut quid possit, non quid actum sit quaerere.

\marginnote{9} verum tamen nimiae vestrae benignitati pareremus, si alia ratione ius nostrum recuperare possemus. nunc vero quis est qui aut vim hominibus armatis factam relinqui putet oportere aut eius rei leviorem actionem nobis aliquam demonstrare possit? ex quo genere peccati, ut illi clamitant, vel iniuriarum vel capitis iudicia constituta sunt, in eo potestis atrocitatem nostram reprehendere, cum videatis nihil aliud actum nisi possessionem per interdictum esse repetitam? verum, sive vos existimationis illius periculum sive iuris dubitatio tardiores fecit adhuc ad iudicandum, alterius rei causam vosmet ipsi iam vobis saepius prolato iudicio sustulistis, alterius ego vobis hodierno die causam profecto auferam, ne diutius de controversia nostra ac de communi iure dubitetis.

\marginnote{10} et si forte videbor altius initium rei demonstrandae petisse quam me ratio iuris eius de quo iudicium est et natura causae coegerit, quaeso ut ignoscatis. non enim minus laborat A.~Caecina ne summo iure egisse quam ne certum ius non obtinuisse videatur.

M.~Fulcinius fuit, recuperatores, e municipio Tarquiniensi; qui et domi suae cum primis honestus existimatus est et Romae argentariam non ignobilem fecit. is habuit in matrimonio Caesenniam, eodem e municipio summo loco natam et probatissimam feminam, sicut et vivus ipse multis rebus ostendit et in morte sua testamento declaravit.
\marginnote{11} huic Caesenniae fundum in agro Tarquiniensi vendidit temporibus illis difficillimis solutionis; cum uteretur uxoris dote numerata, quo mulieri res esset cautior, curavit ut in eo fundo dos conlocaretur. aliquanto post iam argentaria dissoluta Fulcinius huic fundo uxoris continentia quaedam praedia atque adiuncta mercatur. moritur Fulcinius — multa enim, quae sunt in re, quia remota sunt a causa, praetermittam — testamento facit heredem quem habebat e Caesennia filium; usum et fructum omnium bonorum suorum Caesenniae legat ut frueretur una cum filio. \marginnote{12} Magnus honos viri iucundus mulieri fuisset, si diuturnum esse licuisset; frueretur enim bonis cum eo quem suis bonis heredem esse cupiebat et ex quo maximum fructum ipsa capiebat. sed hunc fructum mature fortuna ademit. nam brevi tempore M.~Fulcinius adulescens mortuus est; heredem P.~Caesennium fecit; uxori grande pondus argenti matrique partem maiorem bonorum legavit. itaque in partem mulieres vocatae sunt.

\marginnote{13} Cum esset haec auctio hereditaria constituta, Aebutius iste, qui iam diu Caesenniae viduitate ac solitudine aleretur ac se in eius familiaritatem insinuasset, hac ratione ut cum aliquo suo compendio negotia mulieris, si qua acciderent, controversiasque susciperet, versabatur eo quoque tempore in his rationibus auctionis et partitionis atque etiam se ipse inferebat et intro dabat et in eam opinionem Caesenniam adducebat ut mulier imperita nihil putaret agi callide posse, ubi non adesset Aebutius.
\marginnote{14} quam personam iam ex cotidiana vita cognostis, recuperatores, mulierum adsentatoris, cognitoris viduarum, defensoris nimium litigiosi, contriti ad regiam, inepti ac stulti inter viros, inter mulieres periti iuris et callidi, hanc personam imponite Aebutio. is enim Caesenniae fuit Aebutius — ne forte quaeratis, num propinquus? — nihil alienius — amicus a patre aut a viro traditus?  — nihil minus — quis igitur? ille, ille quem supra deformavi, voluntarius amicus mulieris non necessitudine aliqua, sed ficto officio simulataque sedulitate coniunctus magis opportuna opera non numquam quam aliquando fideli.

\marginnote{15} Cum esset, ut dicere institueram, constituta auctio Romae, suadebant amici cognatique Caesenniae, id quod ipsi quoque mulieri veniebat in mentem, quoniam potestas esset emendi fundum illum Fulcinianum, qui fundo eius antiquo continens esset, nullam esse rationem amittere eius modi occasionem, cum ei praesertim pecunia ex partitione deberetur; nusquam posse eam melius conlocari. itaque hoc mulier facere constituit; mandat ut fundum sibi emat, — cui tandem? — cui putatis? an non in mentem vobis venit omnibus illius hoc munus esse ad omnia mulieris negotia parati, sine quo nihil satis caute, nihil satis callide posset agi? \marginnote{16} recte attenditis. Aebutio negotium datur. adest ad tabulam, licetur Aebutius; deterrentur emptores multi partim gratia Caesenniae, partim etiam pretio. fundus addicitur Aebutio; pecuniam argentario promittit Aebutius; quo testimonio nunc vir optimus utitur sibi emptum esse. quasi vero aut nos ei negemus addictum aut tum quisquam fuerit qui dubitaret quin emeretur Caesenniae, cum id plerique scirent, omnes fere audissent, qui non audisset, is coniectura adsequi posset, cum pecunia Caesenniae ex illa hereditate deberetur, eam porro in praediis conlocari maxime expediret, essent autem praedia quae mulieri maxime convenirent, ea venirent, liceretur is quem Caesenniae dare operam nemo miraretur, sibi emere nemo posset suspicari. \marginnote{17} hac emptione facta pecunia solvitur a Caesennia; cuius rei putat iste rationem reddi non posse quod ipse tabulas averterit; se autem habere argentarii tabulas in quibus sibi expensa pecunia lata sit acceptaque relata. quasi id aliter fieri oportuerit. Cum omnia ita facta essent, quem ad modum nos defendimus, Caesennia fundum possedit locavitque; neque ita multo post A.~Caecinae nupsit. Vt in pauca conferam, testamento facto mulier moritur; facit heredem ex deunce et semuncia Caecinam, ex duabus sextulis M.~Fulcinium, libertum superioris viri, Aebutio sextulam aspergit. hanc sextulam illa mercedem isti esse voluit adsiduitatis et molestiae si quam ceperat. iste autem hac sextula se ansam retinere omnium controversiarum putat.

\marginnote{18} iam principio ausus est dicere non posse heredem esse Caesenniae Caecinam, quod is deteriore iure esset quam ceteri cives propter incommodum Volaterranorum calamitatemque civilem. itaque homo timidus imperitusque, qui neque animi neque consili satis haberet, non putavit esse tanti hereditatem ut de civitate in dubium veniret; concessit, credo, Aebutio, quantum vellet de Caesenniae bonis ut haberet. immo, ut viro forti ac sapienti dignum fuit, ita calumniam stultitiamque eius obtrivit ac contudit. \marginnote{19} in possessione bonorum cum esset, et cum iste sextulam suam nimium exaggeraret, nomine heredis arbitrum familiae herciscundae postulavit. atque illis paucis diebus, postea quam videt nihil se ab A.~Caecina posse litium terrore abradere, homini Romae in foro denuntiat fundum illum de quo ante dixi, cuius istum emptorem demonstravi fuisse mandatu Caesenniae, suum esse seseque sibi emisse. quid ais? istius ille fundus est quem sine ulla controversia quadriennium, hoc est ex quo tempore fundus veniit, quoad vixit, possedit Caesennia? »Vsus enim,« inquit, »eius fundi et fructus testamento viri fuerat Caesenniae.«

\marginnote{20} Cum hoc novae litis genus tam malitiose intenderet, placuit Caecinae de amicorum sententia constituere, quo die in rem praesentem veniretur et de fundo Caecina moribus deduceretur. conloquuntur; dies ex utriusque commodo sumitur. Caecina cum amicis ad diem venit in castellum Axiam, a quo loco fundus is de quo agitur non longe abest. ibi certior fit a pluribus homines permultos liberos atque servos coegisse et armasse Aebutium. Cum id partim mirarentur, partim non crederent, ecce ipse Aebutius in castellum venit; denuntiat Caecinae se armatos habere; abiturum eum non esse, si accessisset. Caecinae placuit et amicis, quoad videretur salvo capite fieri posse, experiri tamen. \marginnote{21} de castello descendunt, in fundum proficiscuntur. videtur temere commissum, verum, ut opinor, hoc fuit causae: tam temere istum re commissurum quam verbis minitabatur nemo putavit. atque iste ad omnis introitus qua adiri poterat non modo in eum fundum de quo erat controversia, sed etiam in illum proximum de quo nihil ambigebatur armatos homines opponit. itaque primo cum in antiquum fundum ingredi vellet, quod ea proxime accedi poterat, frequentes armati obstiterunt. \marginnote{22} quo loco depulsus Caecina tamen qua potuit ad eum fundum profectus est in quo ex conventu vim fieri oportebat; eius autem fundi extremam partem oleae derecto ordine definiunt. ad eas cum accederetur, iste cum omnibus copiis praesto fuit servumque suum nomine Antiochum ad se vocavit et voce clara imperavit ut eum qui illum olearum ordinem intrasset occideret. homo mea sententia prudentissimus Caecina tamen in hac re plus mihi animi quam consili videtur habuisse. nam cum et armatorum multitudinem videret et eam vocem Aebuti quam commemoravi audisset, tamen accessit propius et iam ingrediens intra finem eius loci quem oleae terminabant impetum armati Antiochi ceterorumque tela atque incursum refugit. eodem tempore se in fugam conferunt amici advocatique eius metu perterriti, quem ad modum illorum testem dicere audistis. \marginnote{23} his rebus ita gestis P. Dolabella praetor interdixit, ut est consuetudo, DE VI HOMINIBVS ARMATIS sine ulla exceptione, tantum ut unde deiecisset restitueret. restituisse se dixit. sponsio facta est. hac de sponsione vobis iudicandum est.

maxime fuit optandum Caecinae, recuperatores, ut controversiae nihil haberet, secundo loco ut ne cum tam improbo homine, tertio ut cum tam stulto haberet. etenim non minus nos stultitia illius sublevat quam laedit improbitas. improbus fuit, quod homines coegit, armavit, coactis armatisque vim fecit. laesit in eo Caecinam, sublevat ibidem; nam in eas ipsas res quas improbissime fecit testimonia sumpsit et eis in causa testimoniis utitur.

\marginnote{24} itaque mihi certum est, recuperatores, ante quam ad meam defensionem meosque testis venio, illius uti confessione et testimoniis; qui confitetur atque ita libenter confitetur ut non solum fateri sed etiam profiteri videatur, recuperatores: »convocavi homines, coegi, armavi, terrore mortis ac periculo capitis ne accederes obstiti; ferro,« inquit, »ferro« — et hoc dicit in iudicio — »te reieci atque proterrui.« quid? testes quid aiunt? P.~Vetilius, propinquus Aebuti, se Aebutio cum armatis servis venisse advocatum. quid praeterea? fuisse compluris armatos. quid aliud? minatum esse Aebutium Caecinae. quid ego de hoc teste dicam nisi hoc, recuperatores, ut ne idcirco minus ei credatis quod homo minus idoneus habetur, sed ideo credatis quod ex illa parte id dicit quod illi causae maxime sit alienum? \marginnote{25} A.~Terentius, alter testis, non modo Aebutium sed etiam se pessimi facinoris arguit. in Aebutium hoc dicit, armatos homines fuisse, de se autem hoc praedicat, Antiocho, Aebuti servo, se imperasse ut in Caecinam advenientem cum ferro invaderet. quid loquar amplius de hoc homine? in quem ego hoc dicere, cum rogarer a Caecina, numquam volui, ne arguere illum rei capitalis viderer, de eo dubito nunc quo modo aut loquar aut taceam, cum ipse hoc de se iuratus praedicet. \marginnote{26} deinde L.~Caelius non solum Aebutium cum armatis dixit fuisse compluribus verum etiam cum advocatis perpaucis eo venisse Caecinam. de hoc ego teste detraham? cui aeque ac meo testi ut credatis postulo. P.~Memmius secutus est qui suum non parvum beneficium commemoravit in amicos Caecinae, quibus sese viam per fratris sui fundum dedisse dixit qua effugere possent, cum essent omnes metu perterriti. huic ego testi gratias agam, quod et in re misericordem se praebuerit et in testimonio religiosum. \marginnote{27} A.~Atilius et eius filius L.~Atilius et armatos ibi fuisse et se suos servos adduxisse dixerunt; etiam hoc amplius: cum Aebutius Caecinae malum minaretur, ibi tum Caecinam postulasse ut moribus deductio fieret. hoc idem P.~Rutilius dixit, et eo libentius dixit ut aliquando in iudicio eius testimonio creditum putaretur. duo praeterea testes nihil de vi, sed de re ipsa atque emptione fundi dixerunt; P.~Caesennius, auctor fundi, non tam auctoritate gravi quam corpore, et argentarius Sex.~Clodius cui cognomen est Phormio, nec minus niger nec minus confidens quam ille Terentianus est Phormio, nihil de vi dixerunt, nihil praeterea quod ad vestrum iudicium pertineret.

\marginnote{28} Decimo vero loco testis exspectatus et ad extremum reservatus dixit, senator populi Romani, splendor ordinis, decus atque ornamentum iudiciorum, exemplar antiquae religionis, Fidiculanius Falcula; qui cum ita vehemens acerque venisset ut non modo Caecinam periurio suo laederet sed etiam mihi videretur irasci, ita eum placidum mollemque reddidi, ut non auderet, sicut meministis, iterum dicere quot milia fundus suus abesset ab urbe. nam cum dixisset minus iↄↄↄ, populus cum risu adclamavit ipsa esse. meminerant enim omnes quantum in Albiano iudicio accepisset. \marginnote{29} in eum quid dicam nisi id quod negare non possit, venisse in consilium publicae quaestionis, cum eius consili iudex non esset, et in eo consilio, cum causam non audisset et potestas esset ampliandi, dixisse sibi liquere; cum de incognita re iudicare voluisset, maluisse condemnare quam absolvere; cum, si uno minus damnarent, condemnari reus non posset, non ad cognoscendam causam sed ad explendam damnationem praesto fuisse? Vtrum gravius aliquid in quempiam dici potest quam ad hominem condemnandum quem numquam vidisset neque audisset adductum esse pretio? an certius quicquam obici potest quam quod is cui obicitur ne nutu quidem infirmare conatur? \marginnote{30} verum tamen is testis, — ut facile intellegeretis eum non adfuisse animo, cum causa ab illis ageretur testesque dicerent, sed tantisper de aliquo reo cogitasse — cum omnes ante eum dixissent testes armatos cum Aebutio fuisse compluris, solus dixit non fuisse. visus est mihi primo veterator intellegere praeclare quid causae obstaret, et tantum modo errare, quod omnis testis infirmaret qui ante eum dixissent: cum subito, ecce idem qui solet, duos solos servos armatos fuisse dixit. quid huic tu homini facias? nonne concedas interdum ut excusatione summae stultitiae summae improbitatis odium deprecetur?

\marginnote{31} Vtrum, recuperatores, his testibus non credidistis, cum quid liqueret non habuistis? at controversia non erat quin verum dicerent. an in coacta multitudine, in armis, in telis, in praesenti metu mortis perspicuoque periculo caedis dubium vobis fuit inesse vis aliqua videretur necne? quibus igitur in rebus vis intellegi potest, si in his non intellegetur? an vero illa defensio vobis praeclara visa est: »non deieci, sed obstiti; non enim sum passus in fundum ingredi, sed armatos homines opposui, ut intellegeres, si in fundo pedem posuisses, statim tibi esse pereundum?« quid ais? is qui armis proterritus, fugatus, pulsus est, non videtur esse deiectus? \marginnote{32} posterius de verbo videbimus; nunc rem ipsam ponamus quam illi non negant et eius rei ius actionemque quaeramus.

est haec res posita quae ab adversario non negatur, Caecinam, cum ad constitutam diem tempusque venisset ut vis ac deductio moribus fieret, pulsum prohibitumque esse vi coactis hominibus et armatis. Cum hoc constet, ego, homo imperitus iuris, ignarus negotiorum ac litium, hanc puto me habere actionem, ut per interdictum meum ius teneam atque iniuriam tuam persequar. fac in hoc errare me nec ullo modo posse per hoc interdictum id adsequi quod velim; te uti in hac re magistro volo.

\marginnote{33} quaero sitne aliqua huius rei actio an nulla. convocari homines propter possessionis controversiam non oportet, armari multitudinem iuris retinendi causa non convenit; nec iuri quicquam tam inimicum quam vis nec aequitati quicquam tam infestum est quam convocati homines et armati. quod cum ita sit resque eius modi sit ut in primis a magistratibus animadvertenda videatur, iterum quaero sitne eius rei aliqua actio an nulla. nullam esse dices? audire cupio, qui in pace et otio, cum manum fecerit, copias pararit, multitudinem hominum coegerit, armarit, instruxerit, homines inermos qui ad constitutum experiendi iuris gratia venissent armis, viris, terrore periculoque mortis reppulerit, fugarit, averterit, hoc dicat:
\marginnote{34} »feci equidem quae dicis omnia, et ea sunt et turbulenta et temeraria et periculosa. quid ergo est? impune feci; nam quid agas mecum ex iure civili ac praetorio non habes.«

itane vero? recuperatores, hoc vos audietis et apud vos dici patiemini saepius? Cum maiores nostri tanta diligentia prudentiaque fuerint ut omnia omnium non modo tantarum rerum sed etiam tenuissimarum iura statuerint persecutique sint, hoc genus unum vel maximum praetermitterent, ut, si qui me exire domo mea coegisset armis, haberem actionem, si qui introire prohibuisset, non haberem? nondum de Caecinae causa disputo, nondum de iure possessionis nostrae loquor; tantum de tua defensione, C.~Piso, quaero. \marginnote{35} quoniam ita dicis et ita constituis, si Caecina, cum in fundo esset, inde deiectus esset, tum per hoc interdictum eum restitui oportuisse; nunc vero deiectum nullo modo esse inde ubi non fuerit; hoc interdicto nihil nos adsecutos esse: quaero, si te hodie domum tuam redeuntem coacti homines et armati non modo limine tectoque aedium tuarum sed primo aditu vestibuloque prohibuerint, quid acturus sis. monet amicus meus te, L.~Calpurnius, ut idem dicas quod ipse antea dixit, iniuriarum. quid ad causam possessionis, quid ad restituendum eum quem oportet restitui, quid denique ad ius civile, aut ad praetoris notionem atque animadversionem? ages iniuriarum. plus tibi ego largiar; non solum egeris verum etiam condemnaris licet; num quid magis possidebis? actio enim iniuriarum non ius possessionis adsequitur sed dolorem imminutae libertatis iudicio poenaque mitigat.

\marginnote{36} praetor interea, Piso, tanta de re tacebit? quem ad modum te restituat in aedis tuas non habebit? qui dies totos aut vim fieri vetat aut restitui factam iubet, qui de fossis, de cloacis, de minimis aquarum itinerumque controversiis interdicit, is repente obmutescet, in atrocissima re quid faciat non habebit? et C.~Pisoni domo tectisque suis prohibito, prohibito inquam, per homines coactos et armatos, praetor quem ad modum more et exemplo opitulari possit non habebit? quid enim dicet, aut quid tu tam insigni accepta iniuria postulabis? »Vnde vi prohibitus?« sic nemo umquam interdixit; novum est, non dico inusitatum, verum omnino inauditum. »Vnde deiectus?« quid proficies, cum illi hoc respondebunt tibi quod tu nunc mihi, armatis se tibi obstitisse ne in aedis accederes; deici porro nullo modo potuisse qui non accesserit?

\marginnote{37} »deicior ego,« inquis, »si quis meorum deicitur omnino.« iam bene agis; a verbis enim recedis et aequitate uteris. nam verba quidem ipsa si sequi volumus, quo modo tu deiceris, cum servus tuus deicitur? verum ita est uti dicis; te deiectum debeo intellegere, etiam si tactus non fueris. nonne? age nunc, si ne tuorum quidem quisquam loco motus erit atque omnes in aedibus adservati ac retenti, tu solus prohibitus et a tuis aedibus vi atque armis proterritus, utrum hanc actionem habebis qua nos usi sumus, an aliam quampiam, an omnino nullam? nullam esse actionem dicere in re tam insigni tamque atroci neque prudentiae neque auctoritatis tuae est; alia si quae forte est quae nos fugerit, dic quae sit; cupio discere. \marginnote{38} haec si est qua nos usi sumus te iudice vincamus necesse est. non enim vereor ne hoc dicas, in eadem causa eodem interdicto te oportere restitui, Caecinam non oportere. etenim cui non perspicuum est ad incertum revocari bona, fortunas, possessiones omnium, si ulla ex parte sententia huius interdicti deminuta aut infirmata sit, si auctoritate virorum talium vis armatorum hominum iudicio approbata videatur, in quo iudicio non de armis dubitatum sed de verbis quaesitum esse dicatur? isne apud vos obtinebit causam suam qui se ita defenderit: »reieci ego te armatis hominibus, non deieci,« ut tantum facinus non in aequitate defensionis, sed in una littera latuisse videatur? \marginnote{39} huiusce rei vos statuetis nullam esse actionem, nullum experiendi ius constitutum, qui obstiterit armatis hominibus, qui multitudine coacta non introitu, sed omnino aditu quempiam prohibuerit?

quid ergo? hoc quam habet vim, ut distare aliquid aut ex aliqua parte differre videatur, utrum, pedem cum intulero atque in possessione vestigium fecero, tum expellar ac deiciar, an eadem vi et isdem armis mihi ante occurratur, ne non modo intrare verum aspicere aut aspirare possim? quid hoc ab illo differt, ut ille cogatur restituere qui ingressum expulerit, ille qui ingredientem reppulerit non cogatur?
\marginnote{40} videte, per deos immortalis! quod ius nobis, quam condicionem vobismet ipsis, quam denique civitati legem constituere velitis. huiusce generis una est actio per hoc interdictum quo nos usi sumus constituta; ea si nihil valet aut si ad hanc rem non pertinet, quid neglegentius aut quid stultius maioribus nostris dici potest, qui aut tantae rei praetermiserint actionem aut eam constituerint quae nequaquam satis verbis causam et rationem iuris amplecteretur? hoc est periculosum, dissolvi hoc interdictum, est captiosum omnibus rem ullam constitui eius modi quae, cum armis gesta sit, rescindi iure non possit; verum tamen illud est turpissimum, tantae stultitiae prudentissimos homines condemnari, ut vos iudicetis huius rei ius atque actionem in mentem maioribus nostris non venisse.

\marginnote{41} »queramur,« inquit, »licet; tamen hoc interdicto Aebutius non tenetur.« quid ita? »quod vis Caecinae facta non est.« dici in hac causa potest, ubi arma fuerint, ubi coacta hominum multitudo, ubi instructi et certis locis cum ferro homines conlocati, ubi minae, pericula terroresque mortis, ibi vim non fuisse? »nemo,« inquit, »occisus est neque saucius factus.« quid ais? cum de possessionis controversia et de privatorum hominum contentione iuris loquamur, tu vim negabis factam, si caedes et occisio facta non erit? at exercitus maximos saepe pulsos et fugatos esse dico terrore ipso impetuque hostium sine cuiusquam non modo morte verum etiam volnere.
