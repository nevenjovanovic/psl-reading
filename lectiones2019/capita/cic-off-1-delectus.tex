% https://scaife.perseus.org/reader/urn:cts:latinLit:phi0474.phi055.perseus-lat1:1.1-1.57

\marginnote{1} Quamquam te, Marce fili, annum iam audientem Cratippum, idque Athenis, abundare oportet praeceptis institutisque philosophiae propter summam et doctoris auctoritatem et urbis, quorum alter te scientia augere potest, altera exemplis, tamen, ut ipse ad meam utilitatem semper cum Graecis Latina coniunxi neque id in philosophia solum, sed etiam in dicendi exercitatione feci, idem tibi censeo faciendum, ut par sis in utriusque orationis facultate. Quam quidem ad rem nos, ut videmur, magnum attulimus adiumentum hominibus nostris, ut non modo Graecarum litterarum rudes, sed etiam docti aliquantum se arbitrentur adeptos et ad dicendum et ad iudicandum.

\marginnote{2} Quam ob rem disces tu quidem a principe huius aetatis philosophorum, et disces, quam diu voles; tam diu autem velle debebis, quoad te, quantum proficias, non paenitebit; sed tamen nostra legens non multum a Peripateticis dissidentia, quoniam utrique Socratici et Platonici volumus esse, de rebus ipsis utere tuo iudicio (nihil enim impedio), orationem autem Latinam efficies profecto legendis nostris pleniorem. Nec vero hoc arroganter dictum existimari velim. Nam philosophandi scientiam concedens multis, quod est oratoris proprium, apte, distincte, ornate dicere, quoniam in eo studio aetatem consumpsi, si id mihi assumo, videor id meo iure quodam modo vindicare.

 

\marginnote{3} Quam ob rem magnopere te hortor, mi Cicero, ut non solum orationes meas, sed hos etiam de philosophia libros, qui iam illis fere se aequarunt, studiose legas; vis enim maior in illis dicendi, sed hoc quoque colendum est aequabile et temperatum orationis genus. Et id quidem nemini video Graecorum adhuc contigisse, ut idem utroque in genere elaboraret sequereturque et illud forense dicendi et hoc quietum disputandi genus, nisi forte Demetrius Phalereus in hoc numero haberi potest, disputator subtilis, orator parum vehemens, dulcis tamen, ut Theophrasti discipulum possis agnoscere. Nos autem quantum in utroque profecerimus, aliorum sit iudicium, utrumque certe secuti sumus.
 

\marginnote{4} Equidem et Platonem existimo, si genus forense dicendi tractare voluisset, gravissime et copiosissime potuisse dicere, et Demosthenem, si illa, quae a Platone didicerat, tenuisset et pronuntiare voluisset, ornate splendideque facere potuisse; eodemque modo de Aristotele et Isocrate iudico, quorum uterque suo studio delectatus contempsit alterum.

Sed cum statuissem scribere ad te aliquid hoc tempore, multa posthac, ab eo ordiri maxime volui, quod et aetati tuae esset aptissimum et auctoritati meae. Nam cum multa sint in philosophia et gravia et utilia accurate copioseque a philosophis disputata, latissime patere videntur ea, quae de officiis tradita ab illis et praecepta sunt. Nulla enim vitae pars neque publicis neque privatis neque forensibus neque domesticis in rebus, neque si tecum agas quid, neque si cum altero contrahas, vacare officio potest, in eoque et colendo sita vitae est honestas omnis et neglegendo turpitudo.

\marginnote{5} Atque haec quidem quaestio communis est omnium philosophorum; quis est enim, qui nullis officii praeceptis tradendis philosophum se audeat dicere? Sed sunt non nullae disciplinae, quae propositis bonorum et malorum finibus officium omne pervertant. Nam qui summum bonum sic instituit, ut nihil habeat cum virtute coniunctum, idque suis commodis, non honestate metitur, hic, si sibi ipse consentiat et non interdum naturae bonitate vincatur neque amicitiam colere possit nec iustitiam nec liberalitatem; fortis vero dolorem summum malum iudicans aut temperans voluptatem summum bonum statuens esse certe nullo modo potest.
 

\marginnote{6} Quae quamquam ita sunt in promptu, ut res disputatione non egeat, tamen sunt a nobis alio loco disputata. Hae disciplinae igitur si sibi consentaneae velint esse, de officio nihil queant dicere, neque ulla officii praecepta firma, stabilia, coniuncta naturae tradi possunt nisi aut ab iis, qui solam, aut ab iis, qui maxime honestatem propter se dicant expetendam. Ita propria est ea praeceptio Stoicorum, Academicorum, Peripateticorum, quoniam Aristonis, Pyrrhonis, Erilli iam pridem explosa sententia est; qui tamen haberent ius suum disputandi de officio, si rerum aliquem dilectum reliquissent, ut ad officii inventionem aditus esset. Sequemur igitur hoc quidem tempore et hac in quaestione potissimum Stoicos non ut interpretes, sed, ut solemus, e fontibus eorum iudicio arbitrioque nostro, quantum quoque modo videbitur, hauriemus.

 

\marginnote{7} Placet igitur, quoniam omnis disputatio de officio futura est, ante definire, quid sit officium; quod a Panaetio praetermissum esse miror. Omnis enim, quae ratione suscipitur de aliqua re institutio, debet a definitione proficisci, ut intellegatur, quid sit id, de quo disputetur...

Omnis de officio duplex est quaestio: unum genus est, quod pertinet ad finem bonorum, alterum, quod positum est in praeceptis, quibus in omnis partis usus vitae conformari possit. Superioris generis huius modi sunt exempla: omniane officia perfecta sint, num quod officium aliud alio maius sit, et quae sunt generis eiusdem. Quorum autem officiorum praecepta traduntur, ea quamquam pertinent ad finem bonorum, tamen minus id apparet, quia magis ad institutionem vitae communis spectare videntur; de quibus est nobis his libris explicandum. Atque etiam alia divisio est officii.


\marginnote{8} Nam et medium quoddam officium dicitur et perfectum. Perfectum officium rectum, opinor, vocemus, quoniam Graeci \textgreek{κατόρθωμα}, hoc autem commune officium \textgreek{καθῆκον} vocant. Atque ea sic definiunt, ut, rectum quod sit, id officium perfectum esse definiant; medium autem officium id esse dicunt, quod cur factum sit, ratio probabilis reddi possit.

 

\marginnote{9} Triplex igitur est, ut Panaetio videtur, consilii capiendi deliberatio. Nam aut honestumne factu sit an turpe dubitant id, quod in deliberationem cadit; in quo considerando saepe animi in contraries sententias distrahuntur. Tum autem aut anquirunt aut consultant, ad vitae commoditatem iucunditatemque, ad facultates rerum atque copias, ad opes, ad potentiam, quibus et se possint iuvare et suos, conducat id necne, de quo deliberant; quae deliberatio omnis in rationem utilitatis cadit. Tertium dubitandi genus est, cum pugnare videtur cum honesto id, quod videtur esse utile; cum enim utilitas ad se rapere, honestas contra revocare ad se videtur, fit ut distrahatur in deliberando animus afferatque ancipitem curam cogitandi.
 

\marginnote{10} Hac divisione, cum praeterire aliquid maximum vitium in dividendo sit, duo praetermissa sunt; nec enim solum utrum honestum an turpe sit, deliberari solet, sed etiam duobus propositis honestis utrum honestius, itemque duobus propositis utilibus utrum utilius. Ita, quam ille triplicem putavit esse rationem, in quinque partes distribui debere reperitur. Primum igitur est de honesto, sed dupliciter, tum pari ratione de utili, post de comparatione eorum disserendum.

 

\marginnote{11} Principio generi animantium omni est a natura tributum, ut se, vitam corpusque tueatur, declinet ea, quae nocitura videantur, omniaque, quae sint ad vivendum necessaria, anquirat et paret, ut pastum, ut latibula, ut alia generis eiusdem. Commune item animantium omnium est coniunctionis adpetitus procreandi causa et cura quaedam eorum, quae procreata sint; sed inter hominem et beluam hoc maxime interest, quod haec tantum, quantum sensu movetur, ad id solum, quod adest quodque praesens est, se accommodat paulum admodum sentiens praeteritum aut futurum; homo autem, quod rationis est particeps, per quam consequentia cernit, causas rerum videt earumque praegressus et quasi antecessiones non ignorat, similitudines comparat rebusque praesentibus adiungit atque annectit futuras, facile totius vitae cursum videt ad eamque degendam praeparat res necessarias.

 

\marginnote{12} Eademque natura vi rationis hominem conciliat homini et ad orationis et ad vitae societatem ingeneratque in primis praecipuum quendam amorem in eos, qui procreati sunt, impellitque, ut hominum coetus et celebrationes et esse et a se obiri velit ob easque causas studeat parare ea, quae suppeditent ad cultum et ad victum, nec sibi soli, sed coniugi, liberis ceterisque, quos caros habeat tuerique debeat; quae cura exsuscitat etiam animos et maiores ad rem gerendam facit.

 

\marginnote{13} In primisque hominis est propria veri inquisitio atque investigatio. Itaque cum sumus necessariis negotiis curisque vacui, tum avemus aliquid videre, audire, addiscere cognitionemque rerum aut occultarum aut admirabilium ad beate vivendum necessariam ducimus. Ex quo intellegitur, quod verum, simplex sincerumque sit, id esse naturae hominis aptissimum. Huic veri videndi cupiditati adiuncta est appetitio quaedam principatus, ut nemini parere animus bene informatus a natura velit nisi praecipienti aut docenti aut utilitatis causa iuste et legitime imperanti; ex quo magnitudo animi exsistit humanarumque rerum contemptio.
 

\marginnote{14} Nec vero illa parva vis naturae est rationisque. quod unum hoc animal sentit, quid sit ordo, quid sit, quod deceat, in factis dictisque qui modus. Itaque eorum ipsorum, quae aspectu sentiuntur, nullum aliud animal pulchritudinem, venustatem, convenientiam partium sentit; quam similitudinem natura ratioque ab oculis ad animum transferens multo etiam magis pulchritudinem, constantiam, ordinem in consiliis factisque conservandam putat cavetque, ne quid indecore effeminateve faciat, tum in omnibus et opinionibus et factis ne quid libidinose aut faciat aut cogitet.

Quibus ex rebus conflatur et efficitur id, quod quaerimus, honestum, quod etiamsi nobilitatum non sit, tamen honestum sit, quodque vere dicimus, etiamsi a nullo laudetur, natura esse laudabile.
 

\marginnote{15} Formam quidem ipsam, Marce fili, et tamquam faciem honesti vides, \textit{quae si oculis cerneretur, mirabiles amores,} ut ait Plato, \textit{excitaret sapientiae.} Sed omne, quod est honestum, id quattuor partium oritur ex aliqua: aut enim in perspicientia veri sollertiaque versatur aut in hominum societate tuenda tribuendoque suum cuique et rerum contractarum fide aut in animi excelsi atque invicti magnitudine ac robore aut in omnium, quae fiunt quaeque dicuntur, ordine et modo, in quo inest modestia et temperantia.

Quae quattuor quamquam inter se colligata atque implicata sunt, tamen ex singulis certa officiorum genera nascuntur, velut ex ea parte, quae prima discripta est, in qua sapientiam et prudentiam ponimus, inest indagatio atque inventio veri, eiusque virtutis hoc munus est proprium.


\marginnote{16} Ut enim quisque maxime perspicit, quid in re quaque verissimum sit, quique acutissime et celerrime potest et videre et explicare rationem, is prudentissimus et sapientissimus rite haberi solet. Quocirca huic quasi materia, quam tractet et in qua versetur, subiecta est veritas.

\marginnote{17} Reliquis autem tribus virtutibus necessitates propositae sunt ad eas res parandas tuendasque, quibus actio vitae continetur, ut et societas hominum coniunctioque servetur et animi excellentia magnitudoque cum in augendis opibus utilitatibusque et sibi et suis comparandis, tum multo magis in his ipsis despiciendis eluceat. Ordo autem et constantia et moderatio et ea, quae sunt his similia, versantur in eo genere, ad quod est adhibenda actio quaedam, non solum mentis agitatio. Iis enim rebus, quae tractantur in vita, modum quendam et ordinem adhibentes honestatem et decus conservabimus.
 

\marginnote{18} Ex quattuor autem locis, in quos honesti naturam vimque divisimus, primus ille, qui in veri cognitione consistit, maxime naturam attingit humanam. Omnes enim trahimur et ducimur ad cognitionis et scientiae cupiditatem, in qua excellere pulchrum putamus, labi autem, errare, nescire, decipi et malum et turpe ducimus. In hoc genere et naturali et honesto duo vitia vitanda sunt, unum, ne incognitapro cognitis habeamus iisque temere assentiamur; quod vitium effugere qui volet (omnes autem velle debent), adhibebit ad considerandas res et tempus et diligentiam.


\marginnote{19} Alterum est vitium, quod quidam nimis magnum studium multamque operam in res obscuras atque difficiles conferunt easdemque non necessarias.

Quibus vitiis declinatis quod in rebus honestis et cognitione dignis operae curaeque ponetur, id iure laudabitur, ut in astrologia C.~Sulpicium audivimus, in geometria Sex.~Pompeium ipsi cognovimus, multos in dialecticis, plures in iure civili, quae omnes artes in veri investigatione versantur; cuius studio a rebus gerendis abduci contra officium est. Virtutis enim laus omnis in actione consistit; a qua tamen fit intermissio saepe multique dantur ad studia reditus; tum agitatio mentis, quae numquam acquiescit, potest nos in studiis cognitionis etiam sine opera nostra continere. Omnis autem cogitatio motusque animi aut in consiliis capiendis de rebus honestis et pertinentibus ad bene beateque vivendum aut in studiis scientiae cognitionisque versabitur.
Ac de primo quidem officii fonte diximus.
 

\marginnote{20} De tribus autem reliquis latissime patet ea ratio, qua societas hominum inter ipsos et vitae quasi communitas continetur; cuius partes duae, iustitia, in qua virtutis est splendor maximus, ex qua viri boni nominantur, et huic coniuncta beneficentia, quam eandem vel benignitatem vel liberalitatem appellari licet.

Sed iustitiae primum munus est, ut ne cui quis noceat nisi lacessitus iniuria, deinde ut communibus pro communibus utatur, privatis ut suis.
 

\marginnote{21} Sunt autem privata nulla natura, sed aut vetere occupatione, ut qui quondam in vacua venerunt, aut victoria, ut qui bello potiti sunt, aut lege, pactione, condicione, sorte; ex quo fit, ut ager Arpinas Arpinatium dicatur, Tusculanus Tusculanorum; similisque est privatarum possessionum discriptio. Ex quo, quia suum cuiusque fit eorum, quae natura fuerant communia, quod cuique obtigit, id quisque teneat; e quo si quis sibi appetet, violabit ius humanae societatis.
 
\marginnote{22} Sed quoniam, ut praeclare scriptum est a Platone, non nobis solum nati sumus ortusque nostri partem patria vindicat, partem amici, atque, ut placet Stoicis, quae in terris gignantur, ad usum hominum omnia creari, homines autem hominum causa esse generatos, ut ipsi inter se aliis alii prodesse possent, in hoc naturam debemus ducem sequi, communes utilitates in medium afferre mutatione officiorum, dando accipiendo, tum artibus, tum opera, tum facultatibus devincire hominum inter homines societatem.

\marginnote{23} Fundamentum autem est iustitiae fides, id est dictorum conventorumque constantia et veritas. Ex quo, quamquam hoc videbitur fortasse cuipiam durius, tamen audeamus imitari Stoicos, qui studiose exquirunt, unde verba sint ducta, credamusque, quia fiat, quod dictum est, appellatam fidem.

Sed iniustitiae genera duo sunt, unum eorum, qui inferunt, alterum eorum, qui ab iis, quibus infertur, si possunt, non propulsant iniuriam. Nam qui iniuste impetum in quempiam facit aut ira aut aliqua perturbatione incitatus, is quasi manus afferre videtur socio; qui autem non defendit nec obsistit, si potest, iniuriae, tam est in vitio, quam si parentes aut amicos aut patriam deserat. 

\marginnote{24} Atque illae quidem iniuriae, quae nocendi causa de industria inferuntur, saepe a metu proficiscuntur, cum is, qui nocere alteri cogitat, timet ne, nisi id fecerit, ipse aliquo afficiatur incommodo. Maximam autem partem ad iniuriam faciendam aggrediuntur, ut adipiscantur ea, quae concupiverunt; in quo vitio latissime patet avaritia.
 

\marginnote{25} Expetuntur autem divitiae cum ad usus vitae necessarios, tum ad perfruendas voluptates. In quibus autem maior est animus, in iis pecuniae cupiditas spectat ad opes et ad gratificandi facultatem, ut nuper M.~Crassus negabat ullam satis magnam pecuniam esse ei, qui in re publica princeps vellet esse, cuius fructibus exercitum alere non posset. Delectant etiam magnifici apparatus vitaeque cultus cum elegantia et copia; quibus rebus effectum est, ut infinita pecuniae cupiditas esset. Nec vero rei familiaris amplificatio nemini nocens vituperanda est, sed fugienda semper iniuria est.

 

\marginnote{26} Maxime autem adducuntur plerique, ut eos iustitiae capiat oblivio, cum in imperiorum, honorum, gloriae cupiditatem inciderunt. Quod enim est apud Ennium:
\begin{verse}
Núlla sancta sócietas\\
Néc fides regni ést.
\end{verse}
id latius patet. Nam quicquid eius modi est, in quo non possint plures excellere, in eo fit plerumque tanta contentio, ut difficillimum sit servare sanctam societatem.

Declaravit id modo temeritas C.~Caesaris, qui omnia iura divina et humana pervertit propter eum, quem sibi ipse opinionis errore finxerat, principatum. Est autem in hoc genere molestum, quod in maximis animis splendidissimisque ingeniis plerumque exsistunt honoris, imperii, potentiae, gloriae cupiditates. Quo magis cavendum est, ne quid in eo genere peccetur.


\marginnote{27} Sed in omni iniustitia permultum interest, utrum perturbatione aliqua animi, quae plerumque brevis est et ad tempus, an consulto et cogitata fiat iniuria. Leviora enim sunt ea, quae repentino aliquo motu accidunt, quam ea, quae meditata et praeparata inferuntur.

Ac de inferenda quidem iniuria satis dictum est.
 

\marginnote{28} Praetermittendae autem defensionis deserendique officii plures solent esse causae; nam aut inimicitias aut laborem aut sumptus suscipere nolunt aut etiam neglegentia, pigritia, inertia aut suis studiis quibusdam occupationibusve sic impediuntur, ut eos, quos tutari debeant, desertos esse patiantur. Itaque videndum est, ne non satis sit id, quod apud Platonem est in philosophos dictum, quod in veri investigatione versentur quodque ea, quae plerique vehementer expetant, de quibus inter se digladiari soleant, contemnant et pro nihilo putent, propterea iustos esse. Nam alterum assequuntur, ut inferenda ne cui noceant iniuria, in alterum incidunt; discendi enim studio impediti, quos tueri debent, deserunt. Itaque eos ne ad rem publicam quidem accessuros putat nisi coactos. Aequius autem erat id voluntate fieri; namhoc ipsum ita iustum est, quod recte fit, si est voluntarium.

 

\marginnote{29} Sunt etiam, qui aut studio rei familiaris tuendae aut odio quodam hominum suum se negotium agere dicant nec facere cuiquam videantur iniuriam. Qui altero genere iniustitiae vacant, in alterum incurrunt; deserunt enim vitae societatem, quia nihil conferunt in eam studii, nihil operae, nihil facultatum.

Quando igitur duobus generibus iniustitiae propositis adiunximus causas utriusque generis easque res ante constituimus, quibus iustitia contineretur, facile, quod cuiusque temporis officium sit, poterimus, nisi nosmet ipsos valde amabimus, iudicare; \marginnote{30} est enim difficilis cura rerum alienarum. Quamquam Terentianus ille Chremes \textit{humani nihil a se alienum putat}; sed tamen, quia magis ea percipimus atque sentimus, quae nobis ipsis aut prospera aut adversa eveniunt, quam illa, quae ceteris, quae quasi longo intervallo interiecto videmus, aliter de illis ac de nobis iudicamus. Quocirca bene praecipiunt, qui vetant quicquam agere, quod dubites aequum sit an iniquum. Aequitas enim lucet ipsa per se, dubitatio cogitationem significat iniuriae.

 

\marginnote{31} Sed incidunt saepe tempora, cum ea, quae maxime videntur digna esse iusto homine eoque, quem virum bonum dicimus, commutantur fiuntque contraria, ut reddere depositum, facere promissum quaeque pertinent ad veritatem et ad fidem, ea migrare interdum et non servare fit iustum. Referri enim decet ad ea, quae posui principio, fundamenta iustitiae, primum ut ne cui noceatur, deinde ut communi utilitati serviatur. Ea cum tempore commutantur, commutatur officium et non semper est idem.


\marginnote{32} Potest enim accidere promissum aliquod et conventum, ut id effici sit inutile vel ei, cui promissum sit, vel ei, qui promiserit. Nam si, ut in fabulis est, Neptunus, quod Theseo promiserat, non fecisset, Theseus Hippolyto filio non esset orbatus; ex tribus enim optatis, ut scribitur, hoc erat tertium, quod de Hippolyti interitu iratus optavit; quo impetrato in maximos luctus incidit. Nec promissa igitur servanda sunt ea, quae sint iis, quibus promiseris, inutilia, nec, si plus tibi ea noceant quam illi prosint, cui promiseris, contra officium est maius anteponi minori; ut, si constitueris cuipiam te advocatum in rem praesentem esse venturum atque interim graviter aegrotare filius coeperit, non sit contra officium non facere, quod dixeris, magisque ille, cui promissum sit, ab officio discedat, si se destitutum queratur. Iam illis promissis standum non esse quis non videt, quae coactus quis metu, quae deceptus dolo promiserit? quae quidem pleraque iure praetorio liberantur, non nulla legibus.

 

\marginnote{33} Exsistunt etiam saepe iniuriae calumnia quadam et nimis callida, sed malitiosa iuris interpretatione. Ex quo illud \textit{Summum ius summa iniuria} factum est iam tritum sermone proverbium. Quo in genere etiam in re publica multa peccantur, ut ille, qui, cum triginta dierum essent cum hoste indutiae factae, noctu populabatur agros, quod dierum essent pactae, non noctium indutiae. Ne noster quidem probandus, si verum est Q.~Fabium Labeonem seu quem alium (nihil enim habeo praeter auditum) arbitrum Nolanis et Neapolitanis de finibus a senatu datum, cum ad locum venisset, cum utrisque separatim locutum, ne cupide quid agerent, ne appetenter, atque ut regredi quam progredi mallent. Id cum utrique fecissent, aliquantum agri in medio relictum est. Itaque illorum finis sic, ut ipsi dixerant, terminavit; in medio relictum quod erat, populo Romano adiudicavit. Decipere hoc quidem est, non iudicare. Quocirca in omni est re fugienda talis sollertia.


Sunt autem quaedam officia etiam adversus eos servanda, a quibus iniuriam acceperis. Est enim ulciscendi et puniendi modus; atque haud scio an satis sit eum, qui lacessierit, iniuriae suae paenitere, ut et ipse ne quid tale posthac et ceteri sint ad iniuriam tardiores.

 

\marginnote{34} Atque in re publica maxime conservanda sunt iura belli. Nam cum sint duo genera decertandi, unum per disceptationem, alterum per vim, cumque illud proprium sit hominis, hoc beluarum, confugiendum est ad posterius, si uti non licet superiore. 

\marginnote{35} Quare suscipienda quidem bella sunt ob eam causam, ut sine iniuria in pace vivatur, parta autem victoria conservandi ii, qui non crudeles in bello, non immanes fuerunt, ut maiores nostri Tusculanos, Aequos, Volscos, Sabinos, Hernicos in civitatem etiam acceperunt, at Carthaginem et Numantiam funditus sustulerunt; nollem Corinthum, sed credo aliquid secutos, opportunitatem loci maxime, ne posset aliquando ad bellum faciendum locus ipse adhortari. Mea quidem sententia paci, quae nihil habitura sit insidiarum, semper est consulendum. In quo si mihi esset optemperatum, si non optimam, at aliquam rem publicam, quae nunc nulla est, haberemus.

Et cum iis, quos vi deviceris, consulendum est, tum ii, qui armis positis ad imperatorum fidem confugient, quamvis murum aries percusserit, recipiendi. In quo tantopere apud nostros iustitia culta est, ut ii, qui civitates aut nationes devictas bello in fidem recepissent, earum patroni essent more maiorum.

 

\marginnote{36} Ac belli quidem aequitas sanctissime fetiali populi Romani iure perscripta est. Ex quo intellegi potest nullum bellum esse iustum, nisi quod aut rebus repetitis geratur aut denuntiatum ante sit et indictum.

\marginnote{37} M.~quidem Catonis senis est epistula ad M.~filium, in qua scribit se audisse eum missum factum esse a consule, cum in Macedonia bello Persico miles esset. Monet igitur, ut caveat, ne proelium ineat; negat enim ius esse, qui miles non sit, cum hoste pugnare.

Equidem etiam illud animadverto, quod, qui proprio nomine perduellis esset, is hostis vocaretur, lenitate verbi rei tristitiam mitigatam. Hostis enim apud maiores nostros is dicebatur, quem nunc peregrinum dicimus. Indicant duodecim tabulae: \textit{aut status dies cum hoste,} itemque: \textit{adversus hostem aeterna auctoritas.} Quid ad hanc mansuetudinem addi potest, eum, quicum bellum geras, tam molli nomine appellare? Quamquam id nomen durius effect iam vetustas; a peregrino enim recessit et proprie in eo, qui arma contra ferret, remansit.

 

\marginnote{38} Cum vero de imperio decertatur belloque quaeritur gloria, causas omnino subesse tamen oportet easdem, quas dixi paulo ante iustas causas esse bellorum. Sed ea bella, quibus imperii proposita gloria est, minus acerbe gerenda sunt. Ut enim cum civi aliter contendimus, si est inimicus, aliter, si competitor (cum altero certamen honoris et dignitatis est, cum altero capitis et famae), sic cum Celtiberis, cum Cimbris bellum ut cum inimicis gerebatur, uter esset, non uter imperaret, cum Latinis, Sabinis, Samnitibus, Poenis, Pyrrho de imperio dimicabatur. Poeni foedifragi, crudelis Hannibal, reliqui iustiores. Pyrrhi quidem de captivis reddendis illa praeclara:
\begin{verse}
Nec mi aurum posco nec mi pretium dederitis,\\
Nec cauponantes bellum, sed belligerantes\\
Ferro, non auro vitam cernamus utrique.

Vosne velit an me regnare era, quidve ferat Fors,\\
Virtute experiamur. Et hoc simul accipe dictum:\\
Quorum virtuti belli fortuna pepercit,\\
Eorundem libertati me parcere certum est.\\
Dono, ducite, doque volentibus cum magnis dis.
\end{verse}

Regalis sane et digna Aeacidarum genere sententia.
 

\marginnote{39} Atque etiam si quid singuli temporibus adducti hosti promiserunt, est in eo ipso fides conservanda, ut primo Punico bello Regulus captus a Poenis cum de captivis commutandis Romam missus esset iurassetque se rediturum, primum, ut venit, captivos reddendos in senatu non censuit, deinde, cum retineretur a propinquis et ab amicis, ad supplicium redire maluit quam fidem hosti datam fallere.

\marginnote{41} Ac de bellicis quidem officiis satis dictum est.

Meminerimus autem etiam adversus infimos iustitiam esse servandam. Est autem infima condicio et fortuna servorum, quibus non male praecipiunt qui ita iubent uti, ut mercennariis: operam exigendam, iusta praebenda.

Cum autem duobus modis, id est aut vi aut fraude, fiat iniuria, fraus quasi vulpeculae, vis leonis videtur; utrumque homine alienissimum, sed fraus odio digna maiore. Totius autem iniustitiae nulla capitalior quam eorum, qui tum, cum maxime fallunt, id agunt, ut viri boni esse videantur.

De iustitia satis dictum.
 

\marginnote{42} Deinceps, ut erat propositum, de beneficentia ae de liberalitate dicatur, qua quidem nihil est naturae hominis accommodatius, sed habet multas cautiones. Videndum est enim, primum ne obsit benignitas et iis ipsis, quibus benigne videbitur fieri et ceteris, deinde ne maior benignitas sit quam facultates, tum ut pro dignitate cuique tribuatur; id enim est iustitiae fundamentum, ad quam haec referenda sunt omnia. Nam et qui gratificantur cuipiam, quod obsit illi, cui prodesse velle videantur, non benefici neque liberales, sed perniciosi assentatores iudicandi sunt, et qui aliis nocent, ut in alios liberales sint, in eadem sunt iniustitia, ut si in suam rem aliena convertant.

 

\marginnote{43} Sunt autem multi, et quidem cupidi splendoris et gloriae, qui eripiunt aliis, quod aliis largiantur, iique arbitrantur se beneficos in suos amicos visum iri, si locupletent eos quacumque ratione. Id autem tantum abest ab officio, ut nihil magis officio possit esse contrarium. Videndum est igitur, ut ea liberalitate utamur, quae prosit amicis, noceat nemini. Quare L.~Sullae, C.~Caesaris pecuniarum translatio a iustis dominis ad alienos non debet liberalis videri; nihil est enim liberale, quod non idem iustum.

 

\marginnote{44} Alter locus erat cautionis, ne benignitas maior esset quam facultates, quod, qui benigniores volunt esse, quam res patitur, primum in eo peccant, quod iniuriosi sunt in proximos; quas enim copias his et suppeditari aequius est et relinqui, eas transferunt ad alienos. Inest autem in tali liberalitate cupiditas plerumque rapiendi et auferendi per iniuriam, ut ad largiendum suppetant copiae. Videre etiam licet plerosque non tam natura liberales quam quadam gloria ductos, ut benefici videantur, facere multa, quae proficisci ab ostentatione magis quam a voluntate videantur. Talis autem sinulatio vanitati est coniunctior quam aut liberalitati aut honestati.
 

\marginnote{45} Tertium est propositum, ut in beneficentia dilectus esset dignitatis; in quo et mores eius erunt spectandi, in quem beneficium conferetur, et animus erga nos et communitas ac societas vitae et ad nostras utilitates officia ante collata; quae ut concurrant omnia, optabile est; si minus, plures causae maioresque ponderis plus habebunt.
 

\marginnote{46} Quoniam autem vivitur non cum perfectis hominibus planeque sapientibus, sed cum iis, in quibus praeclare agitur si sunt simulacra virtutis, etiam hoc intellegendum puto, neminem omnino esse neglegendum, in quo aliqua significatio virtutis appareat, colendum autem esse ita quemque maxime, ut quisque maxime virtutibus his lenioribus erit ornatus, modestia, temperantia, hac ipsa, de qua multa iam dicta sunt, iustitia. Nam fortis animus et magnus in homine non perfecto nec sapiente ferventior plerumque est, illae virtutes bonum virum videntur potius attingere.

Atque haec in moribus.
 

\marginnote{47} De benivolentia autem, quam quisque habeat erga nos, primum illud est in officio, ut ei plurimum tribuamus, a quo plurimum diligamur, sed benivolentiam non adulescentulorum more ardore quodam amoris, sed stabilitate potius et constantia iudicemus. Sin erunt merita, ut non ineunda, sed referenda sit gratia, maior quaedam cura adhibenda est; nullum enim officium referenda gratia magis necessarium est.
 

\marginnote{48} Quodsi ea, quae utenda acceperis, maiore mensura, si modo possis, iubet reddere Hesiodus, quidnam beneficio provocati facere debemus? an imitari agros fertiles, qui multo plus efferunt quam acceperunt? Etenim si in eos, quos speramus nobis profuturos, non dubitamus officia conferre, quales in eos esse debemus, qui iam profuerunt? Nam cum duo genera liberalitatis sint, unum dandi beneficii, alterum reddendi, demus necne, in nostra potestate est, non reddere viro bono non licet, modo id facere possit sine iniuria.

 

\marginnote{49} Acceptorum autem beneficiorum sunt dilectus habendi, nec dubium, quin maximo cuique plurimum debeatur. In quo tamen in primis, quo quisque animo, studio, benivolentia fecerit, ponderandum est. Multi enim faciunt multa temeritate quadam sine iudicio vel morbo in omnes vel repentino quodam quasi vento impetu animi incitati; quae beneficia aeque magna non sunt habenda atque ea, quae iudicio, considerate constanterque delata sunt.

Sed in collocando beneficio et in referenda gratia, si cetera paria sunt, hoc maxime officii est, ut quisque maxime opis indigeat, ita ei potissimum opitulari; quod contra fit a plerisque; a quo enim plurimum sperant, etiamsi ille iis non eget, tamen ei potissimum inserviunt.
 

\marginnote{50} Optime autem societas hominum coniunctioque servabitur, si, ut quisque erit coniunctissimus, ita in eum benignitatis plurimum conferetur.


Sed, quae naturae principia sint communitatis et societatis humanae, repetendum videtur altius; est enim primum, quod cernitur in universi generis humani societate. Eius autem vinculum est ratio et oratio, quae docendo, discendo, communicando, disceptando, iudicando conciliat inter se homines coniungitque naturali quadam societate; neque ulla re longius absumus a natura ferarum, in quibus inesse fortitudinem saepe dicimus, ut in equis, in leonibus, iustitiam, aequitatem, bonitatem non dicimus; sunt enim rationis et orationis expertes.

\marginnote{51} Ac latissime quidem patens hominibus inter ipsos, omnibus inter omnes societas haec est; in qua omnium rerum, quas ad communem hominum usum natura genuit, est servanda communitas, ut, quae discripta sunt legibus et iure civili, haec ita teneantur, ut sit constitutum legibus ipsis, cetera sic observentur, ut in Graecorum proverbio est, amicorum esse communia omnia. Omnium autem communia hominum videntur ea, quae sunt generis eius, quod ab Ennio positum in una re transferri in permultas potest:
\begin{verse}
Homó, qui erranti cómiter monstrát viam,\\
Quasi lúmen de suo lúmine accendát, facit.\\
Nihiló minus ipsi lúcet, cum illi accénderit.\\
\end{verse}
Una ex re satis praecipit, ut, quicquid sine detrimento commodari possit, id tribuatur vel ignoto; \marginnote{52} ex quo sunt illa communia: non prohibere aqua profluente, pati ab igne ignem capere, si qui velit, consilium fidele deliberanti dare, quae sunt iis utilia, qui accipiunt, danti non molesta. Quare et his utendum est et semper aliquid ad communem utilitatem afferendum. Sed quoniam copiae parvae singulorum sunt, eorum autem, qui his egeant, infinita est multitudo, vulgaris liberalitas referenda est ad illum Ennii finem: \textit{Nihilo minus ipsi lucet,} ut facultas sit, qua in nostros simus liberales.


\marginnote{53} Gradus autem plures sunt societatis hominum. Ut enim ab illa infinita discedatur, propior est eiusdem gentis, nationis, linguae, qua maxime homines coniunguntur; interius etiam est eiusdem esse civitatis; multa enim sunt civibus inter se communia, forum, fana, porticus, viae, leges, iura: iudicia, suffragia, consuetudines praeterea et familiaritates multisque cum multis res rationesque contractae.

Artior vero colligatio est societatis propinquorum; ab illa enim immensa societate humani generis in exiguum angustumque concluditur. 

\marginnote{54} Nam cum sit hoc natura commune animantium, ut habeant libidinem procreandi, prima societas in ipso coniugio est, proxima in liberis, deinde una domus, communia omnia; id autem est principium urbis et quasi seminarium rei publicae. Sequuntur fratrum coniunctiones, post consobrinorum sobrinorumque, qui cum una domo iam capi non possint, in alias domos tamquam in colonias exeunt. Sequuntur conubia et affinitates, ex quibus etiam plures propinqui; quae propagatio et suboles origo est rerum publicarum. Sanguinis autem coniunctio et benivolentia devincit homines et caritate; \marginnote{55} magnum est enim eadem habere monumenta maiorum, eisdem uti sacris, sepulcra habere communia.

Sed omnium societatum nulla praestantior est, nulla firmior, quam cum viri boni moribus similes sunt familiaritate coniuncti; illud enim honestum quod saepe dicimus, etiam si in alio cernimus, nos movet atque illi, in quo id inesse videtur, amicos facit.


\marginnote{56} Et quamquam omnis virtus nos ad se allicit facitque, ut eos diligamus, in quibus ipsa inesse videatur, tamen iustitia et liberalitas id maxime efficit. Nihil autem est amabilius nec copulatius quam morum similitudo bonorum; in quibus enim eadem studia sunt, eaedem voluntates, in iis fit ut aeque quisque altero delectetur ac se ipso, efficiturque id, quod Pythagoras vult in amicitia, ut unus fiat ex pluribus.

Magna etiam illa communitas est, quae conficitur ex beneficiis ultro et citro datis acceptis, quae et mutua et grata dum sunt, inter quos ea sunt, firma devinciuntur societate.
 

\marginnote{57} Sed cum omnia ratione animoque lustraris, omnium societatum nulla est gravior, nulla carior quam ea, quae cum re publica est uni cuique nostrum. Cari sunt parentes, cari liberi, propinqui, familiars, sed omnes omnium caritates patria una complexa est, pro qua quis bonus dubitet mortem oppetere, si ei sit profuturus? Quo est detestabilior istorum immanitas, qui lacerarunt omni scelere patriam et in ea funditus delenda occupati et sunt et fuerunt.

